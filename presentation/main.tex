%%%%%%%%%%%%%%%%%%%%%%%%%%%%%%%%%%%%%%%%%%%%%%%%%%%%%%%%%%%%%%%%%%%%%%%%%%%%%%%
% Copyright (C)  2015 Philipp Hacker
% Permission is granted to copy, distribute and/or modify this document
% under the terms of the GNU Free Documentation License, Version 1.3
% or any later version published by the Free Software Foundation;
% with no Invariant Sections, no Front-Cover Texts, and no Back-Cover Texts.
% A copy of the license should come with this file and/or can be obtained at
% http://www.gnu.org/licenses/fdl-1.3.html
%%%%%%%%%%%%%%%%%%%%%%%%%%%%%%%%%%%%%%%%%%%%%%%%%%%%%%%%%%%%%%%%%%%%%%%%%%%%%%%

\documentclass{beamer}
\usetheme{UniGreifswald}

\usepackage[ngerman]{babel}
\usepackage[T1]{fontenc}
\usepackage[utf8]{inputenc}

\usepackage[autostyle=true]{csquotes}
\usepackage[backend=bibtex,natbib=true]{biblatex}
% Use the bibtex backend with the authoryear citation style
\addbibresource{bibcontents/preamble.bib} % The filename of the bibliography
\addbibresource{bibcontents/master.bib} % The filename of the bibliography

\usepackage{mathtools}
\usepackage{graphicx}
\usepackage{units}
\usepackage{siunitx}
\usepackage{caption}
\captionsetup{labelformat=empty,labelsep=none}

\newcommand{\diff}{\textnormal{d}}
\newcommand{\tenpo}[1]{\cdot 10^{#1}}
\newcommand{\ix}[1]{_\text{#1}}
\newcommand{\imag}{\mathbf{i}}
\newcommand{\fett}[1]{\textbf{#1}}
\newcommand{\stichpunkt}[1]{\begin{itemize}\item #1\end{itemize}}

\renewcommand*{\bibfont}{\scriptsize}

%%%%%%%%%%%%%%%%%%%%%%%%%%%%%%%%%%%%%%%%%%%%%%%%%%%%%%%%%%%%%%%%%%%%%%%%%%%%%%%
% BLOCKTEMPLATE
% 
% \onslide<->{
% 	\begin{frame}{}{}
% 		\begin{block}{}
% 		\onslide<->{}	
% 		\end{block}
% 	\end{frame}
% }
%%%%%%%%%%%%%%%%%%%%%%%%%%%%%%%%%%%%%%%%%%%%%%%%%%%%%%%%%%%%%%%%%%%%%%%%%%%%%%%
\title{Kinetic\ Effects\ in\ RF\ Discharges}

% \subtitle{}

\author[P. Hacker]{Philipp\ Hacker}

\date{\today}

\institute[Uni Greifswald]% (optional)
{%
	Mathematisch-Naturwissenschaftliche\ Fakultät\\%
	Institut\ für\ Physik\\%
  	Ernst-Moritz-Arndt-Universität\ Greifswald%
}

\begin{document}
%
%
%
%
% Title-Folie mit Betreuer/Gutachter:
	\begin{frame}
		\maketitle%
		\centering%
		{\scriptsize Betreuer:\ Prof.\ Dr.\ R.\ Schneider}\\%
		{\scriptsize Gutachter:\ Prof.\ Dr.\ J.\ Meichsner}%
	\end{frame}
%	
% Folie mit Inhaltsverzeichnis:
	\frame{\tableofcontents}
%
%
%
%
% Motivations-Sektion:
	\section{Motivation}
%
% Kapazitiv gekoppelte RF-Plasmen:
		\begin{frame}{Kapazitive gekopplte RF-Plasmen}
			\begin{columns}
				\column{0.45\textwidth}
					\begin{block}
						\onslide<1-3>{%
							\stichpunkt{Anwendung in Halbleiter- %
							und Computerchip-Industrie}%
						}%
						\onslide<2-3>{%
							\stichpunkt{in elektronegativen %
							CCRF-Entladungen treffen schnelle Ionen %
							auf die Elektroden}%
						}%
						\onslide<3>{%
							\stichpunkt{Oberflächenprozesse an der %
							Elektrode mit negativen Ionen}%
						}%
					\end{block}
				\column{0.45\textwidth}
					\onslide<1-3>{%
						\begin{figure}%
							\includegraphics[width=\textwidth]%
											{figures/niondist_material.png}
							\vspace*{0.5cm}%
							\caption{{\scriptsize\raggedleft%
									  (Negative Ionen Energieverteilung in Sauerstoffentladungen)~%
									  \cite{Scheuer15}}}%
						\end{figure}%
					}%
			\end{columns}%
		\end{frame}%
%
% Randschichteffekte:
		\nologoinbg%
		\begin{frame}{Randschichteffekte}%
			\begin{columns}
				\begin{column}{0.45\textwidth}%
				 	\begin{figure}%
						\includegraphics[width=\textwidth]%
										{figures/circuitselfbias_1.png}%
						\vspace*{0.1cm}
						\caption{{\scriptsize\raggedright%
								  (Schema einer Entladung)~%
								  \cite{Piel10}}}%
					\end{figure}%
				\end{column}%
				\begin{column}{0.45\textwidth}%
					\begin{block}%
						\onslide<1-3>{%
							\stichpunkt{negative Aufladung der Wände durch schnellere Elektronen\newline%
										$\rightarrow$Self-Bias}%
						}%
						\onslide<2-3>{%
							\stichpunkt{Ionen werden auf Bohm-Geschwindigkeit beschleunigt\newline%
										$v\ix{i,B}=\sqrt{\frac{k\ix{B}T\ix{e}}{m\ix{i}}}$}%
						}%
						\onslide<3>{%
							\stichpunkt{Asymmetrie der getriebenen/geerden Elektroden}%
						}%
					\end{block}%
				\end{column}%
			\end{columns}%
		\end{frame}%
		\begin{frame}{Randschichteffekte}%
			\begin{columns}%
				\begin{column}{0.45\textwidth}%
 					\begin{figure}%
						\includegraphics[width=\textwidth]%
										{figures/sheath_piel.png}%
						\vspace*{0.25cm}%
						\caption{{\scriptsize\raggedright%
								  (Dichte und Potential vor einer Wand)~%
								 \cite{Piel10}}}%
					\end{figure}%
				\end{column}%
				\begin{column}{0.45\textwidth}%
					\begin{block}%
						\onslide<1>{%
							\stichpunkt{\dots}
							\stichpunkt{Kapazitive Kopplung führt zur Verschiebung des Plasma-Potentials}%
						}%
					\end{block}%
				\end{column}%
			\end{columns}%
		\end{frame}%
%
% Oberflaechen- und Stossprozesse:
		\begin{frame}{Oberflächen- und Stoßprozesse}%
			\begin{figure}%
				\includegraphics[width=0.8\textwidth]%
								{figures/xsections.png}%
				\caption{{\scriptsize%
						(ausgewählte Stoßquerschnitte in Sauerstoff)%
						}}%
			\end{figure}%
		\end{frame}
%
		\begin{frame}{Oberflächen- und Stoßprozesse}%
			\begin{columns}
				\begin{column}{0.48\textwidth}
					\begin{figure}
						\includegraphics[width=\textwidth]%
										{figures/xsections.png}%
					\end{figure}
				\end{column}
				\begin{column}{0.48\textwidth}
					\begin{figure}
						\includegraphics[width=\textwidth]%
										{figures/tabel_crosssections.png}%
					\end{figure}
				\end{column}
			\end{columns}
		\end{frame}
		\logoinbg%
%
% Das Experiment:
	\section{Experiment}
%
		\begin{frame}{Das Experiment}%
			\begin{columns}%
				\begin{column}{0.48\textwidth}%
					\begin{block}%
						\onslide<1->{%
							\stichpunkt{große Asymmetrie zwischen geerdeter~%
													Kammer und CCRF-Elektrode}%
						}%
						\onslide<2->{%
							\stichpunkt{niedrige Gasflüsse und -drücke~%
							($\le$\SI{5}{sc\centi\meter},\SI{15}{\pascal})}%
						}%
						\onslide<3->{%
							\stichpunkt{Elektrodenabstand~%
													$\sim$\SI{5}{\centi\meter}}%
						}%							
					\end{block}%	
				\end{column}%
				\begin{column}{0.48\textwidth}%
					\begin{figure}%
						\centering
						\includegraphics[width=\textwidth]%
														{figures/chamber_exp.pdf}%
						\caption{(Draufsicht des Experimentes)~\cite{Scheuer15}}%
					\end{figure}%	
				\end{column}%
			\end{columns}%
		\end{frame}%
%
		\begin{frame}
%
%
%
%
% Particle-in-Cell Simulation:
	\section{Particle-in-Cell Methode}
%
% Particle-in-Cell Simulation:
		\begin{frame}{Particle-in-Cell Methode}
		\end{frame}
%
% Monte-Carlo-Collisions:
		\begin{frame}{Monte-Carlo Stoßroutinen}
		\end{frame}
%
%
%
%
% Vergleich mit 1D Simulationen:
	\section{1D Simulation}
%
% Ergebnisse von 1D:
		\begin{frame}{1D Simulation}
		\end{frame}
%
% Energieverteilungen:
		\begin{frame}{Energieverteilungen}
		\end{frame}
%
%
% Dynamik negativer Ionen:
		\begin{frame}{Dynamik negativer Ionen}
		\end{frame}
%
%
%
%
% 2D Simulationen:
	\section{Simulationen in 2D}
%
		\begin{frame}{Simulationen in 2D}
		\end{frame}
%
% Vergleich mit 1D:
		\begin{frame}{Vergleich mit 1D}
		\end{frame}
%
% Negative Ionen EVF:
		\begin{frame}{Negative Ionen EVF}
		\end{frame}
%
% Asymmetrische Randbedingungen:
		\begin{frame}{Asymmetrische Ranbedingungen}
		\end{frame}
%
% Einfluss des Self Bias:
		\begin{frame}{Einfluss des Self Bias}
		\end{frame}
%
%
%
%
% Ausblick:
	\section{Ausblick}
%
		\begin{frame}{Ausblick}
		\end{frame}
%
%
% Referenzen:
	\section{Referenzen}
%
		\begin{frame}{Referenzen}
		\end{frame}
%
%
\end{document}
