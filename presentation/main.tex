%%%%%%%%%%%%%%%%%%%%%%%%%%%%%%%%%%%%%%%%%%%%%%%%%%%%%%%%%%%%%%%%%%%%%%%%%%%%%%%
% Copyright (C)  2015 Philipp Hacker
% Permission is granted to copy, distribute and/or modify this document
% under the terms of the GNU Free Documentation License, Version 1.3
% or any later version published by the Free Software Foundation;
% with no Invariant Sections, no Front-Cover Texts, and no Back-Cover Texts.
% A copy of the license should come with this file and/or can be obtained at
% http://www.gnu.org/licenses/fdl-1.3.html
%%%%%%%%%%%%%%%%%%%%%%%%%%%%%%%%%%%%%%%%%%%%%%%%%%%%%%%%%%%%%%%%%%%%%%%%%%%%%%%

\documentclass{beamer}
%\usetheme{UniGreifswald}
\usetheme{Warsaw}
\beamertemplatenavigationsymbolsempty
\setbeamertemplate{footline}[page number]

\usepackage[ngerman]{babel}
\usepackage[T1]{fontenc}
\usepackage[utf8]{inputenc}

\usepackage[autostyle=true]{csquotes}
\usepackage[%
%			style=authoryear,%
			style=verbose,%
			autocite=footnote,%
			maxbibnames=15,%
			maxcitenames=10,%
%			babel,%
%			hyperref=true,%
			abbreviate=false,%
			backend=bibtex,%
%			mcite%
%			labelyear
]{biblatex}

\addbibresource{bibcontents/preamble.bib} % The filename of the bibliography
\addbibresource{bibcontents/master.bib} % The filename of the bibliography

\usepackage{mathtools}
\usepackage{graphicx}
\usepackage{units}
\usepackage{siunitx}
\usepackage{caption}
\usepackage{subcaption}
\captionsetup{labelformat=empty,labelsep=none}

\newcommand{\diff}{\text{d}}
\newcommand{\tenpo}[1]{\cdot 10^{#1}}
\newcommand{\ix}[1]{_\text{#1}}
\newcommand{\imag}{\mathbf{i}}
\newcommand{\fett}[1]{\textbf{#1}}
\newcommand{\stichpunkt}[1]{\begin{itemize}\item #1\end{itemize}}
\renewcommand*{\bibfont}{\scriptsize}

\title{Kinetic\ Effects\ in\ RF\ Discharges}
\author[P. Hacker]{Philipp\ Hacker}
\date{08.12.2017}
\institute[Uni Greifswald]% (optional)
{%
	Mathematisch-Naturwissenschaftliche\ Fakultät\\%
	Institut\ für\ Physik\\%
  	Ernst-Moritz-Arndt-Universität\ Greifswald%
}

\begin{document}
%
% Title-Folie mit Betreuer/Gutachter:
	\begin{frame}
		\maketitle%
		\centering%
		{\scriptsize Betreuer:\ Prof.\ Dr.\ R.\ Schneider}\\%
		{\scriptsize Gutachter:\ Prof.\ Dr.\ J.\ Meichsner}%
	\end{frame}
%	
% Folie mit Inhaltsverzeichnis:
	\frame{\tableofcontents}
%
% Motivations-Sektion:
	\section{Motivation}
%
% Kapazitiv gekoppelte RF-Plasmen:
		\begin{frame}{Kapazitive gekopplte RF-Plasmen}
			\begin{columns}
				\begin{column}{0.45\textwidth}
					\begin{block}{}
						\stichpunkt{Anwendung in Technologie-Industrie bei Sputter- und Ätzprozessen}%
						\stichpunkt{Energieverteilungen der Plasmaspezies besonders wichtig}%
						\stichpunkt{schnelle Anionen in CCRF-Entladungen}%
					\end{block}
				\end{column}
				\begin{column}{0.45\textwidth}
					\begin{figure}%
						\includegraphics[width=\textwidth]%
										{figures/circuitselfbias_1.png}%
						\caption{{\scriptsize%
								(Schema einer Entladung\footnotemark)}}%
					\end{figure}%
				\end{column}%
			\end{columns}%
			\footcitetext{Piel10}%
		\end{frame}%
	
		\begin{frame}{Experiment}%
			\begin{columns}
				\begin{column}{0.5\textwidth}
					\begin{figure}%
						\centering
						\includegraphics[width=1.1\textwidth]%
										{figures/niondist_material.png}%
						\caption*{{\scriptsize%
									(EVF negativer Ionen, Experiment\footnotemark)}}%
					\end{figure}%
				\end{column}
				\footcitetext{Matthias15}
				\begin{column}{0.45\textwidth}
					\begin{block}{}
						\stichpunkt{asymmetrische Niederdruckentladung in Sauerstoff\\%
									$\Rightarrow$ Self-Bias}
						\stichpunkt{hauptsächlich Produktion der O$^{-}$ durch diss. Anlagerung}
						\stichpunkt{Vermutung: Bildung negativer Ionen an der Kathodenoberfläche}
					\end{block}
				\end{column}
			\end{columns}
		\end{frame}%	
%
% Particle-in-Cell Simulation:
	\section{Particle-in-Cell Methode}
%
% Particle-in-Cell Simulation:
		\begin{frame}{Particle-in-Cell Methode}%
			\begin{figure}%
				\centering%
				\includegraphics[width=0.75\textwidth]%
								{figures/picscheme.pdf}
				\caption*{{\scriptsize%
							(PIC-Zyklus\footnotemark)}}
				\footcitetext{Matthias15}
			\end{figure}
			\vspace{-1.0cm}
			\begin{block}{Stabilitätsbedingungen}
				\centering$%
				\Delta t\le\frac{\omega\ix{p,e}}{5},%
				\quad\quad\quad%
				\Delta r\le\frac{\lambda\ix{D,e}}{2}%
				$
			\end{block}	
		\end{frame}
%
% Boltzmann-Gleichung:
			\begin{frame}{Particle-in-Cell Methode}
				\begin{block}{Boltzmann-Gleichung}
					\raggedright%
					Lösen in jedem Schritt:
					\begin{align*}
						\frac{\partial f\ix{j}}{\partial t}+\vec{v}\cdot\nabla_{\vec{r}}\,f\ix{j}%
						+&\frac{q\ix{j}}{m\ix{j}}\vec{E}\cdot\nabla_{\vec{v}}\,f\ix{j}%
						=\left(\frac{\partial f\ix{j}}{\partial t}\right)\ix{Coll}%
						\onslide<2>{%
							\\&\Bigg\Downarrow%
							Linearisierung%
							\\%
							f\ix{j}(x,v,t+\Delta t)=%
								(1+\Delta& t\cdot I)(1+\Delta t\cdot D)f\ix{j}(x,v,t)%
						}
					\end{align*}
				\end{block}
				\onslide<2>{\begin{alertblock}{}%
					wenige Stöße und große freie Weglängen\\%
					$\Rightarrow$ kinetische Simulation notwendig%
				\end{alertblock}}
			\end{frame}
%		
% Vergleich mit 1D Simulationen:  
	\section{1D Simulation}
%
% Ergebnisse von 1D:
		\begin{frame}{1D Simulation}%
			\begin{block}{}
			\end{block}
			\begin{figure}
				\centering%
				\includegraphics[height=0.5\textheight]%
								{figures/results/1D/densities.png}
				\caption*{{\scriptsize%
							(Dichten einer Entladung bei \SI{800}{\volt}%
							und \SI{5}{\pascal}\footnotemark)}}%
			\end{figure}
			\footcitetext{Matthias17}
%			\begin{figure}%
%				\centering%
%				\begin{subfigure}{0.43\textwidth}%
%					\includegraphics[width=\textwidth]%
%									{figures/results/1D/prs_ne_dens.png}%
%				\end{subfigure}%
%				\begin{subfigure}{0.43\textwidth}%
%					\includegraphics[width=\textwidth]%
%									{figures/results/1D/densities_Omin.png}%
%				\end{subfigure}%
%				\caption{(Entladung bei~%
%						  \SI{5}{\pascal} und \SI{400}{\volt})}%
%			\end{figure}%
		\end{frame}%
%                   	
% Energieverteilungen:
		\begin{frame}{Dynamik negativer Ionen}%
		\end{frame}%
%                   	
		\begin{frame}{Dynamik negativer Ionen}%
			\begin{figure}%
				\centering%
				\includegraphics[height=0.75\textheight]%
								{figures/results/1D/snix_allpi_edf.png}%
					\caption*{{\scriptsize%
								(Phasenaufgelöster Anodenbereich der O$^{-}$)}}%
			\end{figure}%
		\end{frame}%
%
%		\begin{frame}{Experiment-Vergleich}%
%			\begin{figure}%
%				\centering%
%				\begin{subfigure}{0.47\textwidth}
%					\includegraphics[width=\textwidth]%
%									{figures/neg_mg.png}%
%					\caption*{{\scriptsize%
%							(Experiment)}}%
%				\end{subfigure}
%				\begin{subfigure}{0.47\textwidth}
%					\includegraphics[width=\textwidth]%
%									{figures/results/1D/power_energy_cuts.png}%
%					\caption*{{\scriptsize%
%							(Simulation)}}%
%					\end{subfigure}
%			\end{figure}%
%			\begin{block}{}
%				\only<1>{%
%					\stichpunkt{bisher: Vermutung für hochenergetische Struktur in O$^{-}$ EVF %
%								im Vergleich zu Experiment bestätigt}}
%				\only<2>{%
%					\stichpunkt{Spannungsvariation zeigt Abhängigkeit der Energieverteilung %
%								der Oberflächenprozesse}}
%				\only<3>{%
%				\stichpunkt{niederergetische Struktur nicht in Experiment zu finden}}
%			\end{block}
%		\end{frame}%           	
%                   	
% 2D Simulationen:  	
%	\section{Simulationen in 2D}
%                   	
%		\begin{frame}{Simulationen in 2D}
%			\begin{exampleblock}{}
%				\onslide<1-3>{%
%					\stichpunkt{Annahme einer zylinder-symmetrische Entladung um Mitte %
%								der Elektrode}}
%				\onslide<2-3>{%
%					\stichpunkt{jetzt: verschiedene Kombinationen von Randbedingungen,\linebreak%
%								zBsp. Dielektrika}}
%				\onslide<3>{%
%					\stichpunkt{viel größerer numerischer Aufwand}}
%			\end{exampleblock}
%			\begin{figure}%
%				\centering%
%				\begin{subfigure}{0.47\textwidth}
%					\centering
%					\includegraphics[height=0.35\textheight]%
%						{figures/domain_slice.png}%
%				\end{subfigure}
%				\begin{subfigure}{0.47\textwidth}
%					\centering
%					\includegraphics[height=0.35\textheight]%
%						{figures/radial_cylinder.png}%
%					\end{subfigure}
%			\end{figure}%
%		\end{frame} 	
%
% Vergleich mit 1D:
%		\begin{frame}{Vergleich mit 1D}%
%			\begin{figure}%
%				\begin{subfigure}{0.48\textwidth}%
%					\begin{overprint}
%						\centering%
%						\onslide<1>\includegraphics[width=0.85\textwidth]%
%							{figures/results/2D/compare/45365_dens.png}%
%						\onslide<2>\includegraphics[width=0.85\textwidth]%
%							{figures/results/2D/compare/onedtwod_45431_denscompare.png}%
%					\end{overprint}
%				\end{subfigure}
%				\begin{subfigure}{0.48\textwidth}%
%					\begin{overprint}
%						\centering%
%						\onslide<1>\includegraphics[width=0.85\textwidth]%
%							{figures/results/2D/compare/45365prs_pot.png}%
%						\onslide<2>\includegraphics[width=0.85\textwidth]%
%							{figures/results/2D/compare/onedtwod_45431_phicompare.png}%
%					\end{overprint}
%				\end{subfigure}%
%				\caption*{{\scriptsize%
%						   (Dichte und Potential bei \SI{5}{\pascal} und \SI{400}{\volt} %
%				  			im Vergleich)}}%
%			\end{figure}%
%			\begin{alertblock}
%				\only<2>{$\Rightarrow$ globales Flussgleichgewicht verändert Plasma %
%						 um den Gesamtstrom in die Randschichten anzugleichen}
%			\end{alertblock}
%		\end{frame}%
%		
%		\begin{frame}{Vergleich mit 1D}%
%			\begin{figure}%
%				\begin{subfigure}{0.48\textwidth}%
%					\centering%
%					\includegraphics[width=0.75\textwidth]%
%						{figures/results/2D/compare/onedtwod_45431_denscompare.png}%
%				\end{subfigure}
%				\begin{subfigure}{0.48\textwidth}%
%					\centering%
%					\includegraphics[width=0.75\textwidth]%
%						{figures/results/2D/compare/onedtwod_45431_phicompare.png}%
%				\end{subfigure}%
%			\end{figure}%
%			\begin{block}%
%				\onslide<1-3>{\stichpunkt{vergleichbare Profile um die Zylinderachse}}%
%				\onslide<2-3>{\stichpunkt{2D Bulk-Ausdehnung geringer als in 1D}}%
%				\onslide<3>{\stichpunkt{Randschicht und Dichteabfall größer in 2D}}%
%			\end{block}%
%		\end{frame}%
%
%		\begin{frame}%
%			\begin{figure}%
%				\centering%
%				\vspace*{0.7cm}
%				\includegraphics[height=0.9\textheight]%
%								{figures/results/2D/compare/twod_alldensphi_45431.png}%
%			\end{figure}%
%		\end{frame}%
%
% Asymmetrische Randbedingungen:
%		\begin{frame}{Asymmetrische Ranbedingungen}%
%			\begin{columns}%
%				\begin{column}{0.45\textwidth}%
%					\begin{figure}%
%						\centering%
%						\begin{overprint}%
%							\onslide<1>\includegraphics[width=0.85\textwidth]%
%											  			{figures/results/2D/44332/phi.png}%
%							\caption*{{\scriptsize%
%								(Potential, \SI{5}{\pascal}, \SI{500}{\volt},\linebreak%
%								 $U_{sb}=$\SI{-400}{\volt})}}%
%								 \onslide<2>\includegraphics[width=0.85\textwidth]%
%														{figures/results/2D/44332/i_dens.png}%
%							\caption*{{\scriptsize%
%								(Ionendichte, \SI{5}{\pascal},\linebreak%
%								 \SI{400}{\volt}, $U_{sb}=$\SI{-500}{\volt})}}%
%							\onslide<3>\includegraphics[width=0.85\textwidth]%
%														{figures/results/2D/44332/i_velz.png}%
%							\caption*{{\scriptsize%
%								(axiale Ionengeschwindigkeit,\linebreak%
%								 \SI{5}{\pascal}, \SI{400}{\volt}, %
%								 $U_{sb}=$\SI{-500}{\volt})}}%
%						\end{overprint}%
%					\end{figure}%
%				\end{column}%
%				\begin{column}{0.45\textwidth}
%					\begin{block}{}%
%						\onslide<1-3>{%
%							\stichpunkt{linke Elektrode getrieben, restliche %
%										Randbedingungen sind geerdet}}%
%						\onslide<2-3>{%
%							\stichpunkt{Asymmetrie des Bulk und der Randschichten}}%
%						\onslide<3>{%
%							\stichpunkt{größere Beschleunigung bei kleinerer Randschicht}}%
%					\end{block}%
%				\end{column}%
%			\end{columns}%
%		\end{frame}%
%
%		\begin{frame}{Asymmetrische Ranbedingungen}%
%			\begin{columns}%
%				\begin{column}{0.45\textwidth}
%					\begin{block}{}%
%						\onslide<1-3>{%
%							\stichpunkt{beide Elektroden getrieben, obere/linke %
%										Bereiche geerdet und rechts `floatet'\linebreak%
%										$\Rightarrow$ offene Randbedingung}}%
%						\onslide<2-3>{%
%							\stichpunkt{sehr viel höheres Plasmapotential}}%
%						\onslide<3>{%
%							\stichpunkt{ähnliche Randschichten, größer als zuvor geringere Geschwindigkeiten}}%
%					\end{block}%
%				\end{column}%
%				\begin{column}{0.45\textwidth}%
%					\begin{figure}%
%						\centering%
%						\begin{overprint}%
%							\onslide<1>\includegraphics[width=0.85\textwidth]%
%											  			{figures/results/2D/44426/phi.png}%
%							\caption*{{\scriptsize%
%								(Potential, \SI{5}{\pascal}, \SI{400}{\volt},\linebreak%
%								 $U_{sb}=$\SI{-400}{\volt})}}%
%								 \onslide<2>\includegraphics[width=0.85\textwidth]%
%														{figures/results/2D/44426/i_dens.png}%
%							\caption*{{\scriptsize%
%								(Ionendichte, \SI{5}{\pascal},\linebreak%
%								 \SI{400}{\volt}, $U_{sb}=$\SI{-400}{\volt})}}%
%							\onslide<3>\includegraphics[width=0.85\textwidth]%
%														{figures/results/2D/44426/i_velz.png}%
%							\caption*{{\scriptsize%
%								(axiale Ionengeschwindigkeit,\linebreak%
%								 \SI{5}{\pascal}, \SI{400}{\volt}, %
%								 $U_{sb}=$\SI{-400}{\volt})}}%
%						\end{overprint}%
%					\end{figure}%
%				\end{column}%
%			\end{columns}%
%		\end{frame}%
%
% Einfluss des Self Bias:
%		\begin{frame}{Einfluss des Self Bias}
%			\onslide<1->{%
%				\begin{exampleblock}{}%
%						Was für einen Einfluss hat der Self-Bias auf die Dynamik der O$^{-}$ %
%						von der Oberfläche?
%				\end{exampleblock}%
%			}
%			\only<2->{%
%				\begin{figure}%
%					\begin{subfigure}{0.45\textwidth}
%						\centering%
%						\includegraphics[width=0.75\textwidth]%
%							  			{figures/results/2D/SFB/ni_dens_25830000.png}%
%					\end{subfigure}
%					\begin{subfigure}{0.45\textwidth}
%						\centering%
%						\includegraphics[width=0.75\textwidth]%
%							  			{figures/results/2D/SFB/velni_z_25830000.png}%
%					\end{subfigure}
%					\caption*{{\scriptsize%
%								(Ionendichte und axiale Geschwindigkeit einer Entladung %
%								bei \SI{6}{\pascal}, \SI{400}{\volt} und $U_{sb}=$\SI{-200}{\volt})}}%
%				\end{figure}%
%			}
%		\end{frame}
%
%		\begin{frame}{Einfluss des Self-Bias}%
%			\begin{columns}%
%				\begin{column}{0.45\textwidth}%
%					\only<1-3>{%
%						\begin{block}{}%
%							\onslide<1-3>{\stichpunkt{%
%								hohe, axiale Geschwindigkeit der Ionen}}%
%							\onslide<2-3>{\stichpunkt{%
%								abbremsen in der Randschicht bzw. Bulk-Kante}}%
%							\onslide<3>{\stichpunkt{%
%								signifikanter Anteil der an der linken Elektrode erzeugten Teilchen erreicht andere Seite}}%
%						\end{block}
%					}
%					\only<4-5>{%
%						\begin{alertblock}{Erkenntnis}%
%							Die negativen Ionen treffen auf die Anode mit hoher kinetischer Energie aufgrund der zusätzlichen Beschleunigung in der Randschicht durch den Self-Bias.
%						\end{alertblock}
%					}
%					\only<6-7>{%
%						\begin{alertblock}{Erkenntnis}%
%							\onslide<6-7>{\stichpunkt{stoßbehaftete Schicht $\Rightarrow$ kalte Ionen im Bulk}}%
%							\onslide<7>{\stichpunkt{Asymmetrie und Self-Bias notwendig zur Beschreibung der Effekte}}%
%						\end{alertblock}
%					}
%				\end{column}%
%				\begin{column}{0.45\textwidth}%
%					\begin{figure}
%						\begin{overprint}
%							\centering
%							\onslide<1-3>\includegraphics[width=0.8\textwidth]%
%											{figures/results/2D/SFB/velni_z_25830000.png}%
%							\caption*{{\scriptsize%
%								(axiale Ionengeschwindigkeit,\linebreak%
%									\SI{6}{\pascal}, \SI{400}{\volt} und $U_{sb}=$\SI{-200}{\volt})}}%
%							\onslide<4>\includegraphics[height=0.55\textheight]%
%											{figures/results/2D/SFB/ni_distz.png}%
%							\caption*{{\scriptsize%
%								(axiale Ionenenergieverteilung,\linebreak%
%							\SI{6}{\pascal}, \SI{400}{\volt} und $U_{sb}=$\SI{-200}{\volt})}}%
%							\onslide<5>\includegraphics[height=0.5\textheight]%
%											{figures/results/2D/SFB/ni_cut.png}%
%							\caption*{{\scriptsize%
%								(Ionenenergieverteilung an der Anode,\linebreak%
%							\SI{6}{\pascal}, \SI{400}{\volt})}}%
%							\onslide<6-7>\includegraphics[height=0.55\textheight]%
%											{figures/results/2D/SFB/ni_distz.png}%
%							\caption*{{\scriptsize%
%								(axiale Ionenenergieverteilung,\linebreak%
%							\SI{6}{\pascal}, \SI{400}{\volt} und $U_{sb}=$\SI{-200}{\volt})}}%
%						\end{overprint}%
%					\end{figure}%
%				\end{column}%
%			\end{columns}%
%		\end{frame}%
%
%
% Ausblick:
%	\section{Ausblick}
%
%		\begin{frame}{Fazit \& Ausblick}
%			\begin{columns}
%				\begin{column}{0.46\textwidth}
%					\only<1>{\begin{alertblock}{Zusammenfassung}%
%						\stichpunkt{Äquivalenz von axialen Profilen einer 2D-PIC Simulation und einem 1D-Model gezeigt}%
%						\stichpunkt{Asymmetrie und Self-Bias genutzt um Dynamik negativer Ionen in CCRF-Entladungen zu studieren}%
%					\end{alertblock}}
%					\only<2>{\begin{figure}
%						\centering
%						\includegraphics[width=\textwidth]%
%								{figures/results/2D/SFB/power_energy_cuts.png}%
%						\caption*{{\scriptsize%
%								(Simulationsergebnisse, O$^{-}$ Verteilung)}}%
%					\end{figure}}
%				\end{column}
%				\begin{column}{0.46\textwidth}
%					\only<1>{\begin{exampleblock}{Ausblick}%
%						\stichpunkt{Verbesserung durch tatsächliche Stoßquerschnitte und Koeffizienten der Oberflächenprozesse}%
%						\stichpunkt{2D-PIC für andere Modelle, wie zbsp. Sputtering nutzen}%
%					\end{exampleblock}}
%					\only<2>{\begin{figure}
%						\centering
%						\includegraphics[width=\textwidth]%
%								{figures/neg_mg.png}%
%						\caption*{{\scriptsize%
%								(Experimentergebnisse, O$^{-}$-Verteilung)}}%
%					\end{figure}}
%				\end{column}
%			\end{columns}
%		\end{frame}
%
% Referenzen:
		\begin{frame}{Referenzen}
			\printbibliography
		\end{frame}
%
		\begin{frame}
		\end{frame}




% Randschichteffekte:
%		\begin{frame}{Randschicht}%
%			\begin{columns}
%				\begin{column}{0.45\textwidth}%
%				 	\begin{figure}%
%						\includegraphics[width=\textwidth]%
%										{figures/circuitselfbias_1.png}%
%						\caption{{\scriptsize\raggedright%
%								  (Schema einer Entladung\footnotemark)}}%
%					\end{figure}%
%				\end{column}%
%				\footcitetext{Piel10}
%				\begin{column}{0.45\textwidth}%
%					\begin{block}%
%						\stichpunkt{negative Aufladung der Wände durch schnellere Elektronen\newline%
%									$\rightarrow$Self-Bias}%
%						\stichpunkt{Ionen werden auf Bohm-Geschwindigkeit beschleunigt\newline%
%									$v\ix{i,B}=\sqrt{\frac{k\ix{B}T\ix{e}}{m\ix{i}}}$}%
%						\stichpunkt{Asymmetrie der getriebenen/geerden Elektroden}%
%					\end{block}%
%				\end{column}%
%			\end{columns}%
%		\end{frame}%
%
%		\begin{frame}{Randschicht}%
%			\begin{columns}%
%				\begin{column}{0.45\textwidth}%
% 					\begin{figure}%
%						\includegraphics[width=\textwidth]%
%										{figures/sheath_piel.png}%
%						\caption{{\scriptsize%
%								  (Dichte und Potential vor einer Wand\footnotemark)}}%
%					\end{figure}%
%				\end{column}%
%				\footcitetext{Piel10}
%				\begin{column}{0.45\textwidth}%
%					\begin{block}%
%							\stichpunkt{Kapazitive Kopplung führt zur Verschiebung des Plasma-Potentials}%
%					\end{block}%
%				\end{column}%
%			\end{columns}%
%		\end{frame}%
%
% Oberflaechen- und Stossprozesse:
%		\begin{frame}{Oberflächen- und Stoßprozesse}%
%			\begin{figure}%
%				\centering%
%				\includegraphics[height=0.7\textheight]%
%								{figures/xsections_selection.pdf}%
%				\caption{(ausgewählte Stoßquerschnitte in Sauerstoff)}%
%			\end{figure}%
%		\end{frame}%
%
%		\begin{frame}{Experiment}%
%			\begin{columns}%
%				\begin{column}{0.48\textwidth}%
%					\begin{block}%
%						\stichpunkt{große Asymmetrie zwischen geerdeter~%
%									Kammer und CCRF-Elektrode}%
%						\stichpunkt{niedrige Gasflüsse und -drücke~%
%									($\le$\SI{5}{sc\centi\meter}, \SI{15}{\pascal})}%
%						\stichpunkt{Elektrodenabstand~%
%									$\sim$\SI{5}{\centi\meter}}%
%					\end{block}%	
%				\end{column}%
%				\begin{column}{0.48\textwidth}%
%					\begin{figure}%
%						\centering
%						\includegraphics[width=\textwidth]%
%										{figures/chamber_exp.pdf}%
%						\caption{(Experiment-Reaktor, Draufsicht\footnotemark)}}%
%					\end{figure}%	
%				\end{column}%
%				\footcitetext{Scheuer15}
%			\end{columns}%
%		\end{frame}%
%
%		\begin{frame}%
%			\begin{figure}%
%				\includegraphics[width=0.8\textwidth]%
%								{figures/neg_mg.png}%
%			\end{figure}%
%		\end{frame}%
%
%		\begin{frame}{Saha-Langmuir}%
%			\begin{figure}%
%				\centering%
%				\includegraphics[height=0.3\textheight]%
%								{figures/saha_langmuir.pdf}%
%				\caption*{{\scriptsize%
%							(Schema der Oberflächenprozesse negativer %
%							und positiver Ionen\footnotemark)}
%			\end{figure}%
%			\footcitetext{Matthias15}
%			\begin{block}
%				\begin{align*}
%					\alpha^{-}(B^{-})=\frac{(1-r^{-})%
%						\,w^{-}}{(1-r)\,w}\cdot&\exp\left(%
%						\frac{-\overline{\Phi}_{-}+e\sqrt{eV\ix{ext}}+A(B)}{k\ix{B}T}\right)%
%				\end{align*}
%			\end{block}
%		\end{frame}%
%
% Particle-in-Cell Simulation:
%		\begin{frame}{Particle-in-Cell Methode}%
%			\begin{block}{}
%				\centering$%
%				\Delta\Phi=-\frac{\rho}{\varepsilon\ix{0}}%
%				\quad\Rightarrow\quad%
%				\frac{\diff \vec{v}}{\diff t}=-\frac{q}{m}\vec{\nabla}\Phi%
%				\quad\Rightarrow\quad%
%				\frac{\diff \vec{x}}{\diff t}=\vec{v}$
%			\end{block}	
%		\end{frame}
%
% Ergebnisse von 1D:
%		\begin{frame}{1D Simulation}%
%			\begin{figure}%
%				\centering%
%				\begin{subfigure}{0.43\textwidth}%
%					\begin{overprint}%
%						\onslide<1>\includegraphics[width=\textwidth]%
%										{figures/results/1D/densities.png}%
%						\onslide<2>\includegraphics[width=\textwidth]%
%										{figures/results/1D/prs_ne_dens.png}%
%					\end{overprint}%
%				\end{subfigure}%
%				\hspace{0.7cm}%
%				\begin{subfigure}{0.43\textwidth}%
%					\begin{overprint}%
%						\onslide<1>\includegraphics[width=\textwidth]%
%											{figures/results/1D/potential.png}%
%						\onslide<2>\includegraphics[width=\textwidth]%
%											{figures/results/1D/densities_Omin.png}%
%					\end{overprint}%
%				\end{subfigure}%
%				\caption{(Entladung bei~%
%						  \SI{5}{\pascal} und \SI{400}{\volt})}%
%			\end{figure}%
%			\begin{block}%
%				\onslide<1-2>{%
%					Dichte und phasenaufgelöstes Potential in 1D mit~%
%					Injektion negativer Ionen von der Kathode\linebreak%
%					$\Rightarrow \,\, \eta = I($O$^{-}) / %
%					 I($O$^{+}_{2}) = 0,03$%
%				}%
%			\end{block}%
%		\end{frame}%
%
%		\begin{frame}%
%			\begin{figure}%
%				\centering%
%				\begin{subfigure}{0.49\textwidth}%
%					\includegraphics[height=0.6\textheight]%
%									{figures/results/1D/elastcoll.png}%
%				\end{subfigure}%
%				\begin{subfigure}{0.43\textwidth}%
%					\includegraphics[width=\textwidth]%
%									{figures/results/1D/densities_Omin.png}%
%				\end{subfigure}%
%            \end{figure}%
%			\begin{alertblock}%
%					$\Rightarrow$~Schicht nicht vollständig stoßlos\linebreak
%					Dynamik der schnellen negativen Ionen von der gegenueberliegenden Elektrode~%
%					stark durch elastische Stöße mit O$_{2}$ beeinflusst
%			\end{alertblock}%
%		\end{frame}%
%
% Energieverteilungen:
%		\begin{frame}{Dynamik negativer Ionen}%
%			\begin{figure}%
%				\centering%
%				\begin{subfigure}{0.48\textwidth}%
%					\includegraphics[width=\textwidth]%
%									{figures/results/1D/snix_edf.png}%
%					\caption*{{\scriptsize%
%								(Energieverteilung der negativen Ionen von der Oberfläche)}}%
%				\end{subfigure}%
%				\begin{subfigure}{0.48\textwidth}%
%					\begin{overprint}%
%						\onslide<1>\includegraphics[width=\textwidth]%
%													{figures/results/1D/snix_close_edf.png}%
%						\caption*{{\scriptsize%
%									(Anodenbereich der EVF von O$^{-}$)}}%
%						\onslide<2>\includegraphics[width=\textwidth]%
%													{figures/results/1D/snix_surface_edf.png}%
%						\caption*{{\scriptsize%
%						(Surfaceplot der O$^{-}$ EVF)}}%
%						\onslide<3-4>\includegraphics[width=\textwidth]%
%														{figures/results/1D/ix_edf.png}%
%						\caption*{{\scriptsize%
%							(EVF der O$^{+}_{2}$ Ionen)}}%
%						\onslide<5>\includegraphics[width=\textwidth]%
%													{figures/results/1D/ex_edf.png}%
%						\caption*{{\scriptsize%
%									(EVF der Elektronen)}}%
%					\end{overprint}%
%				\end{subfigure}%
%			\end{figure}%
%			\only<1-2>{\begin{alertblock}{}%
%					hochenergetische Struktur bei $\sim$\SI{100}{\electronvolt} %
%					vor Anode, abnehmend zur Kathode mit mittlerer freier %
%					Weglänge der O$^{-}$%
%				\end{alertblock}}%
%			\only<3>{\begin{alertblock}{}%
%					langsame Ionen im Bulk der Entladung\linebreak%
%					$\rightarrow$ schnelle O$^{-}$ abgebremst und kehren in %
%					Randschicht um%
%				\end{alertblock}}%
%			\only<4>{\begin{alertblock}{}\centering%
%					Struktur über großen Energiebereich vor %
%					den Elektroden?\linebreak%
%				\end{alertblock}}
%			\only<5>{\begin{alertblock}{}\centering%
%					Struktur über großen Energiebereich vor %
%					den Elektroden?\linebreak%
%					bei Elektronen keine sichtbar!%
%				\end{alertblock}}%
%		\end{frame}%






















\end{document}
