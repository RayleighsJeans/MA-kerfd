%%%%%%%%%%%%%%%%%%%%%%%%%%%%%%%%%%%%%%%%%%%%%%%%%%%%%%%%%%%%%%%%%%%%%%%%%%%%%%%
% Copyright (C)  2015 Philipp Hacker
% Permission is granted to copy, distribute and/or modify this document
% under the terms of the GNU Free Documentation License, Version 1.3
% or any later version published by the Free Software Foundation;
% with no Invariant Sections, no Front-Cover Texts, and no Back-Cover Texts.
% A copy of the license should come with this file and/or can be obtained at
% http://www.gnu.org/licenses/fdl-1.3.html
%%%%%%%%%%%%%%%%%%%%%%%%%%%%%%%%%%%%%%%%%%%%%%%%%%%%%%%%%%%%%%%%%%%%%%%%%%%%%%%

\documentclass{beamer}
\usetheme{UniGreifswald}

\usepackage[ngerman]{babel}
\usepackage[T1]{fontenc}
\usepackage[utf8]{inputenc}

\usepackage[autostyle=true]{csquotes}
\usepackage[backend=bibtex,natbib=true]{biblatex}
% Use the bibtex backend with the authoryear citation style
\addbibresource{bibcontents/preamble.bib} % The filename of the bibliography
\addbibresource{bibcontents/master.bib} % The filename of the bibliography

\usepackage{mathtools}
\usepackage{graphicx}
\usepackage{units}
\usepackage{siunitx}
\usepackage{caption}
\usepackage{subcaption}
\captionsetup{labelformat=empty,labelsep=none}
\usepackage[normalem]{ulem}

\newcommand{\redout}{\bgroup\markoverwith%
	{\textcolor{red}{\rule[0.5ex]{2pt}{0.8pt}}}\ULon}
\newcommand{\diff}{\text{d}}
\newcommand{\tenpo}[1]{\cdot 10^{#1}}
\newcommand{\ix}[1]{_\text{#1}}
\newcommand{\imag}{\mathbf{i}}
\newcommand{\fett}[1]{\textbf{#1}}
\newcommand{\stichpunkt}[1]{\begin{itemize}\item #1\end{itemize}}

\renewcommand*{\bibfont}{\scriptsize}

%%%%%%%%%%%%%%%%%%%%%%%%%%%%%%%%%%%%%%%%%%%%%%%%%%%%%%%%%%%%%%%%%%%%%%%%%%%%%%%
% BLOCKTEMPLATE
% 
% \onslide<->{
% 	\begin{frame}{}{}
% 		\begin{block}{}
% 		\onslide<->{}	
% 		\end{block}
% 	\end{frame}
% }
%%%%%%%%%%%%%%%%%%%%%%%%%%%%%%%%%%%%%%%%%%%%%%%%%%%%%%%%%%%%%%%%%%%%%%%%%%%%%%%
\title{Kinetic\ Effects\ in\ RF\ Discharges}

% \subtitle{}

\author[P. Hacker]{Philipp\ Hacker}

\date{08.12.2017}

\institute[Uni Greifswald]% (optional)
{%
	Mathematisch-Naturwissenschaftliche\ Fakultät\\%
	Institut\ für\ Physik\\%
  	Ernst-Moritz-Arndt-Universität\ Greifswald%
}

\begin{document}
%
%
%
%
% Title-Folie mit Betreuer/Gutachter:
	\begin{frame}
		\maketitle%
		\centering%
		{\scriptsize Betreuer:\ Prof.\ Dr.\ R.\ Schneider}\\%
		{\scriptsize Gutachter:\ Prof.\ Dr.\ J.\ Meichsner}%
	\end{frame}
%	
% Folie mit Inhaltsverzeichnis:
	\frame{\tableofcontents}
%
%
%
%
% Motivations-Sektion:
	\section{Motivation}
%
% Kapazitiv gekoppelte RF-Plasmen:
		\begin{frame}{Kapazitive gekopplte RF-Plasmen}
			\begin{columns}
				\column{0.45\textwidth}
					\begin{block}
						\onslide<1-3>{%
							\stichpunkt{Anwendung in Halbleiter- %
							und Computerchip-Industrie}%
						}%
						\onslide<2-3>{%
							\stichpunkt{in elektronegativen %
							CCRF-Entladungen treffen schnelle Ionen %
							auf die Elektroden}%
						}%
						\onslide<3>{%
							\stichpunkt{Oberflächenprozesse an der %
							Elektrode mit negativen Ionen}%
						}%
					\end{block}
				\column{0.45\textwidth}
					\onslide<1-3>{%
						\begin{figure}%
							\includegraphics[width=\textwidth]%
											{figures/niondist_material.png}
							\vspace*{0.5cm}%
							\caption{{\scriptsize\raggedleft%
									  (Negative Ionen Energieverteilung in Sauerstoffentladungen)~%
									  \cite{Scheuer15}}}%
						\end{figure}%
					}%
			\end{columns}%
		\end{frame}%
%
% Randschichteffekte:
		\nologoinbg%
		\begin{frame}{Randschichteffekte}%
			\begin{columns}
				\begin{column}{0.45\textwidth}%
				 	\begin{figure}%
						\includegraphics[width=\textwidth]%
										{figures/circuitselfbias_1.png}%
						\vspace*{0.1cm}
						\caption{{\scriptsize\raggedright%
								  (Schema einer Entladung)~%
								  \cite{Piel10}}}%
					\end{figure}%
				\end{column}%
				\begin{column}{0.45\textwidth}%
					\begin{block}%
						\onslide<1-3>{%
							\stichpunkt{negative Aufladung der Wände durch schnellere Elektronen\newline%
										$\rightarrow$Self-Bias}%
						}%
						\onslide<2-3>{%
							\stichpunkt{Ionen werden auf Bohm-Geschwindigkeit beschleunigt\newline%
										$v\ix{i,B}=\sqrt{\frac{k\ix{B}T\ix{e}}{m\ix{i}}}$}%
						}%
						\onslide<3>{%
							\stichpunkt{Asymmetrie der getriebenen/geerden Elektroden}%
						}%
					\end{block}%
				\end{column}%
			\end{columns}%
		\end{frame}%
		\begin{frame}{Randschichteffekte}%
			\begin{columns}%
				\begin{column}{0.45\textwidth}%
 					\begin{figure}%
						\includegraphics[width=\textwidth]%
										{figures/sheath_piel.png}%
						\vspace*{0.25cm}%
						\caption{{\scriptsize\raggedright%
								  (Dichte und Potential vor einer Wand)~%
								 \cite{Piel10}}}%
					\end{figure}%
				\end{column}%
				\begin{column}{0.45\textwidth}%
					\begin{block}%
						\onslide<1>{%
							\stichpunkt{\dots}
							\stichpunkt{Kapazitive Kopplung führt zur Verschiebung des Plasma-Potentials}%
						}%
					\end{block}%
				\end{column}%
			\end{columns}%
		\end{frame}%
%
% Oberflaechen- und Stossprozesse:
		\begin{frame}{Oberflächen- und Stoßprozesse}%
			\begin{figure}%
				\centering%
				\includegraphics[height=0.7\textheight]%
								{figures/xsections_selection.pdf}%
				\caption{(ausgewählte Stoßquerschnitte in Sauerstoff)}%
			\end{figure}%
		\end{frame}%
%
		\logoinbg%
%
% Das Experiment:
	\section{Experiment}
%
		\begin{frame}{Das Experiment}%
			\begin{columns}%
				\begin{column}{0.48\textwidth}%
					\begin{block}%
						\onslide<1-3>{%
							\stichpunkt{große Asymmetrie zwischen geerdeter~%
													Kammer und CCRF-Elektrode}%
						}%
						\onslide<2-3>{%
							\stichpunkt{niedrige Gasflüsse und -drücke~%
							($\le$\SI{5}{sc\centi\meter}, \SI{15}{\pascal})}%
						}%
						\onslide<3>{%
							\stichpunkt{Elektrodenabstand~%
													$\sim$\SI{5}{\centi\meter}}%
						}%							
					\end{block}%	
				\end{column}%
				\begin{column}{0.48\textwidth}%
					\begin{figure}%
						\centering
						\includegraphics[width=\textwidth]%
														{figures/chamber_exp.pdf}%
						\caption{(Draufsicht des Experimentes)~\cite{Scheuer15}}%
					\end{figure}%	
				\end{column}%
			\end{columns}%
		\end{frame}%
%
		\begin{frame}%
			\begin{figure}%
				\begin{subfigure}{0.47\textwidth}%
					\includegraphics[width=\textwidth]%
									{figures/niondist_material.png}%
				\end{subfigure}%
				\begin{subfigure}{0.47\textwidth}%
					\includegraphics[width=\textwidth]%
									{figures/neg_mg.png}%
				\end{subfigure}%
			\end{figure}%
			\begin{block}
				\onslide<2-3>{%
					Struktur in EVF der negativen Ionen in Sauerstoff, Scheuer et.~al\ \cite{Scheuer15}\\%
					Hochenergetische Spitze in Abhängigkeit der Leistung und Material\\%
				}%
				\onslide<3>{%
					$\Rightarrow$ Anionen von der Elektrode?%
				}%
			\end{block}%
		\end{frame}%
%
		\begin{frame}%
			\begin{figure}%
				\centering%
				\includegraphics[height=0.3\textheight]%
								{figures/saha_langmuir.pdf}%
			\end{figure}%
			\begin{block}
				\onslide<2>{%
					Saha-Langmuir Gleichung:\\%
						Ionisation hängt von Austrittsarbeit, Ionisationsenergie, Oberflächentemperatur, %
						Spannung und quantenmechanischen Koeffizienten des Materials ab%
				}%
			\end{block}%
		\end{frame}%
%
		\begin{frame}%
			\begin{figure}%
				\centering%
				\includegraphics[height=0.3\textheight]%
				{figures/saha_langmuir.pdf}%
			\end{figure}%
			\begin{block}
				\onslide<1>{%
					Saha-Langmuir Gleichung:\\%
					\redout{Ionisation hängt von Austrittsarbeit, Ionisationsenergie, Oberflächentemperatur, %
						Spannung und quantenmechanischen Koeffizienten des Materials ab%
					}%
				}%
			\end{block}%
		\end{frame}%
%
%
%
%
% Particle-in-Cell Simulation:
	\section{Particle-in-Cell Methode}
%
% Particle-in-Cell Simulation:
		\begin{frame}{Particle-in-Cell Methode}%
			\begin{columns}%
				\begin{column}{0.45\textwidth}
					\begin{block}
						\onslide<2->{%
%							\begin{align*}
								$\left[1\right]:\,\,%
								\Delta\Phi_{k}\left(r_{i,j},t\ix{k}\right)%
								=-\frac{\rho_{k}%
								\left(r_{i,j}, t_{k}\right)}{\varepsilon_{0}}$%
								\linebreak\linebreak%
%							\end{align*}
						}%
						\onslide<3->{%
%							\begin{align*}%
								$\left[2\right]:\,\,%
								\vec{E}_{k}=-\vec{\nabla}\Phi_{k}$%
								\linebreak\linebreak%
%							\end{align*}%
						}%
						\onslide<4->{%
%							\begin{align*}%
								$\left[3\right]:\,\,%
								\frac{\diff \vec{v}_{k,n}}{\diff t}=%
								\frac{q_{n}}{m_{n}}\vec{E}_{k}%
								\left(r_{n},t_{k}\right)$%
								\linebreak\linebreak%
%							\end{align*}%
						}%
						\onslide<5->{%
%							\begin{align*}%
								$\left[4\right]:\,\,%
								\frac{\diff \vec{x}_{k,n}}{\diff t}=%
								\vec{v}_{k,n}$%
%							\end{align*}%
						}%
					\end{block}
				\end{column}
				\begin{column}{0.55\textwidth}%
					\begin{figure}%
						\centering%
						\includegraphics[width=\textwidth]%
						{figures/picscheme.pdf}
					\end{figure}
				\end{column}
			\end{columns}
			\end{frame}
%
% Vergleich mit 1D Simulationen:  
	\section{1D Simulation}
%
% Ergebnisse von 1D:
		\nologoinbg%
		\begin{frame}{1D Simulation}%
			\begin{figure}%
				\centering%
				\begin{subfigure}{0.43\textwidth}%
					\begin{overprint}%
						\onslide<1>\includegraphics[width=\textwidth]%
										{figures/results/1D/densities.png}%
						\onslide<2>\includegraphics[width=\textwidth]%
										{figures/results/1D/prs_ne_dens.png}%
					\end{overprint}%
				\end{subfigure}%
				\hspace{0.7cm}%
				\begin{subfigure}{0.43\textwidth}%
					\begin{overprint}%
						\onslide<1>\includegraphics[width=\textwidth]%
											{figures/results/1D/potential.png}%
						\onslide<2>\includegraphics[width=\textwidth]%
											{figures/results/1D/densities_Omin.png}%
					\end{overprint}%
				\end{subfigure}%
				\caption{(Entladung bei~%
						  \SI{5}{\pascal} und \SI{400}{\volt})}%
			\end{figure}%
			\begin{block}%
				\onslide<1-2>{%
					Dichte und phasenaufgelöstes Potential in 1D mit~%
					Injektion negativer Ionen von der Kathode\linebreak%
					$\Rightarrow \,\, \eta = I($O$^{-}) / %
					 I($O$_{2}) = 0,03$%
				}%
			\end{block}%
		\end{frame}%
%
		\begin{frame}%
			\begin{figure}%
				\centering%
				\begin{subfigure}{0.49\textwidth}%
					\includegraphics[height=0.6\textheight]%
							{figures/results/1D/elastcoll.png}%
				\end{subfigure}%
				\hspace{-1.0cm}%
				\begin{subfigure}{0.43\textwidth}%
						\includegraphics[width=\textwidth]%
								{figures/results/1D/densities_Omin.png}%
				\end{subfigure}%
            \end{figure}%
			\begin{alertblock}%
				\onslide<2>{%
					$\Rightarrow$~Schicht nicht vollständig stoßlos\linebreak
					Dynamik der schnellen negativen Ionen von der gegenueberliegenden Elektrode~%
					stark durch elastische Stöße mit O$_{2}$ beeinflusst
				}%
			\end{alertblock}%
		\end{frame}%
		\logoinbg%  	
%                   	
% Energieverteilungen:
		\begin{frame}{Dynamik negativer Ionen}%
			\begin{figure}%
				\centering%
				\begin{subfigure}{0.48\textwidth}%
					\includegraphics[width=\textwidth]%
							{figures/results/1D/snix_edf.png}%
					\caption*{{\scriptsize%
						(Energieverteilung der negativen Ionen von der Oberfläche)}}%
				\end{subfigure}%
				\begin{subfigure}{0.48\textwidth}%
					\begin{overprint}%
						\onslide<1>\includegraphics[width=\textwidth]%
										{figures/results/1D/snix_close_edf.png}%
						\caption*{{\scriptsize%
							(Anodenbereich der EVF von O$^{-}$)}}%
						\onslide<2>\includegraphics[width=\textwidth]%
										{figures/results/1D/snix_surface_edf.png}%
						\caption*{{\scriptsize%
						(Surfaceplot der O$^{-}$ EVF)}}%
						\onslide<3-4>\includegraphics[width=\textwidth]%
										{figures/results/1D/ix_edf.png}%
						\caption*{{\scriptsize%
							(EVF der O$^{+}_{2}$ Ionen)}}%
						\onslide<5>\includegraphics[width=\textwidth]%
										{figures/results/1D/ex_edf.png}%
						\caption*{{\scriptsize%
							(EVF der Elektronen)}}%
					\end{overprint}%
				\end{subfigure}%
			\end{figure}%
			\only<1-2>{\begin{alertblock}{}%
					hochenergetische Struktur bei $\sim$\SI{100}{\electronvolt} %
					vor Anode, abnehmend zur Kathode mit mittlerer freier %
					Weglänge der O$^{-}$%
				\end{alertblock}}%
			\only<3>{\begin{alertblock}{}%
					langsame Ionen im Bulk der Entladung\linebreak%
					$\rightarrow$ schnelle O$^{-}$ abgebremst und kehren in %
					Randschicht um%
				\end{alertblock}}%
			\only<4>{\begin{alertblock}{}\centering%
					Struktur über großen Energiebereich vor %
					den Elektroden?\linebreak%
				\end{alertblock}}
			\only<5>{\begin{alertblock}{}\centering%
					Struktur über großen Energiebereich vor %
					den Elektroden?\linebreak%
					bei Elektronen keine sichtbar!%
				\end{alertblock}}%
		\end{frame}%
%                   	
		\begin{frame}{Dynamik negativer Ionen}%
			\begin{figure}%
				\centering%
				\includegraphics[height=0.75\textheight]%
							{figures/results/1D/snix_allpi_edf.png}%
					\caption*{{\scriptsize%
							(Phasenaufgelöster Anodenbereich der O$^{-}$)}}%
			\end{figure}%
		\end{frame}%
%                 
		\nologoinbg
		\begin{frame}{Experiment-Vergleich}%
			\begin{figure}%
				\centering%
				\begin{subfigure}{0.47\textwidth}
					\includegraphics[width=\textwidth]%
									{figures/neg_mg.png}%
					\caption*{{\scriptsize%
							(Experiment)}}%
				\end{subfigure}
				\begin{subfigure}{0.47\textwidth}
					\includegraphics[width=\textwidth]%
									{figures/results/1D/power_energy_cuts.png}%
					\caption*{{\scriptsize%
							(Simulation)}}%
					\end{subfigure}
			\end{figure}%
			\begin{block}{}
				\only<1>{%
					\stichpunkt{bisher: Vermutung für hochenergetische Struktur in O$^{-}$ EVF %
								im Vergleich zu Experiment bestätigt}}
				\only<2>{%
					\stichpunkt{Spannungsvariation zeigt Abhängigkeit der Energieverteilung %
								der Oberflächenprozesse}}
				\only<3>{%
				\stichpunkt{niederergetische Struktur nicht in Experiment zu finden}}
			\end{block}
		\end{frame}%
%                   	
%                   	
% 2D Simulationen:  	
	\section{Simulationen in 2D}
%                   	
		\begin{frame}{Simulationen in 2D}
			\begin{exampleblock}{}
				\onslide<1-3>{%
					\stichpunkt{Annahme einer zylinder-symmetrische Entladung um Mitte %
								der Elektrode}}
				\onslide<2-3>{%
					\stichpunkt{jetzt: verschiedene Kombinationen von Randbedingungen,\linebreak%
								zBsp. Dielektrika}}
				\onslide<3>{%
					\stichpunkt{viel größerer numerischer Aufwand}}
			\end{exampleblock}
			\begin{figure}%
				\centering%
				\begin{subfigure}{0.47\textwidth}
					\centering
					\includegraphics[height=0.35\textheight]%
						{figures/domain_slice.png}%
				\end{subfigure}
				\begin{subfigure}{0.47\textwidth}
					\centering
					\includegraphics[height=0.35\textheight]%
						{figures/radial_cylinder.png}%
					\end{subfigure}
			\end{figure}%
		\end{frame} 	
%
% Vergleich mit 1D:
		\begin{frame}{Vergleich mit 1D}%
			\begin{figure}%
				\begin{subfigure}{0.48\textwidth}%
					\begin{overprint}
						\centering%
						\onslide<1>\includegraphics[width=0.85\textwidth]%
							{figures/results/2D/compare/45365_dens.png}%
						\onslide<2>\includegraphics[width=0.85\textwidth]%
							{figures/results/2D/compare/onedtwod_45431_denscompare.png}%
					\end{overprint}
				\end{subfigure}
				\begin{subfigure}{0.48\textwidth}%
					\begin{overprint}
						\centering%
						\onslide<1>\includegraphics[width=0.85\textwidth]%
							{figures/results/2D/compare/45365prs_pot.png}%
						\onslide<2>\includegraphics[width=0.85\textwidth]%
							{figures/results/2D/compare/onedtwod_45431_phicompare.png}%
					\end{overprint}
				\end{subfigure}%
				\caption*{{\scriptsize%
						   (Dichte und Potential bei \SI{5}{\pascal} und \SI{400}{\volt} %
				  			im Vergleich)}}%
			\end{figure}%
			\begin{alertblock}
				\only<2>{$\Rightarrow$ globales Flussgleichgewicht verändert Plasma %
						 um den Gesamtstrom in die Randschichten anzugleichen}
			\end{alertblock}
		\end{frame}%
%		
		\begin{frame}{Vergleich mit 1D}%
			\begin{figure}%
				\begin{subfigure}{0.48\textwidth}%
					\centering%
					\includegraphics[width=0.75\textwidth]%
						{figures/results/2D/compare/onedtwod_45431_denscompare.png}%
				\end{subfigure}
				\begin{subfigure}{0.48\textwidth}%
					\centering%
					\includegraphics[width=0.75\textwidth]%
						{figures/results/2D/compare/onedtwod_45431_phicompare.png}%
				\end{subfigure}%
			\end{figure}%
			\begin{block}%
				\onslide<1-3>{\stichpunkt{vergleichbare Profile um die Zylinderachse}}%
				\onslide<2-3>{\stichpunkt{2D Bulk-Ausdehnung geringer als in 1D}}%
				\onslide<3>{\stichpunkt{Randschicht und Dichteabfall größer in 2D}}%
			\end{block}%
		\end{frame}%
%
		\begin{frame}%
			\begin{figure}%
				\centering%
				\vspace*{0.7cm}
				\includegraphics[height=0.9\textheight]%
								{figures/results/2D/compare/twod_alldensphi_45431.png}%
			\end{figure}%
		\end{frame}%
		\logoinbg
%
% Asymmetrische Randbedingungen:
		\begin{frame}{Asymmetrische Ranbedingungen}%
			\begin{columns}%
				\begin{column}{0.45\textwidth}%
					\begin{figure}%
						\centering%
						\begin{overprint}%
							\onslide<1>\includegraphics[width=0.85\textwidth]%
											  			{figures/results/2D/44332/phi.png}%
							\caption*{{\scriptsize%
								(Potential, \SI{5}{\pascal}, \SI{500}{\volt},\linebreak%
								 $U_{sb}=$\SI{-400}{\volt})}}%
								 \onslide<2>\includegraphics[width=0.85\textwidth]%
														{figures/results/2D/44332/i_dens.png}%
							\caption*{{\scriptsize%
								(Ionendichte, \SI{5}{\pascal},\linebreak%
								 \SI{400}{\volt}, $U_{sb}=$\SI{-500}{\volt})}}%
							\onslide<3>\includegraphics[width=0.85\textwidth]%
														{figures/results/2D/44332/i_velz.png}%
							\caption*{{\scriptsize%
								(axiale Ionengeschwindigkeit,\linebreak%
								 \SI{5}{\pascal}, \SI{400}{\volt}, %
								 $U_{sb}=$\SI{-500}{\volt})}}%
						\end{overprint}%
					\end{figure}%
				\end{column}%
				\begin{column}{0.45\textwidth}
					\begin{block}{}%
						\onslide<1-3>{%
							\stichpunkt{linke Elektrode getrieben, restliche %
										Randbedingungen sind geerdet}}%
						\onslide<2-3>{%
							\stichpunkt{Asymmetrie des Bulk und der Randschichten}}%
						\onslide<3>{%
							\stichpunkt{größere Beschleunigung bei kleinerer Randschicht}}%
					\end{block}%
				\end{column}%
			\end{columns}%
		\end{frame}%
%
		\begin{frame}{Asymmetrische Ranbedingungen}%
			\begin{columns}%
				\begin{column}{0.45\textwidth}
					\begin{block}{}%
						\onslide<1-3>{%
							\stichpunkt{beide Elektroden getrieben, obere/linke %
										Bereiche geerdet und rechts `floatet'\linebreak%
										$\Rightarrow$ offene Randbedingung}}%
						\onslide<2-3>{%
							\stichpunkt{sehr viel höheres Plasmapotential}}%
						\onslide<3>{%
							\stichpunkt{ähnliche Randschichten, größer als zuvor geringere Geschwindigkeiten}}%
					\end{block}%
				\end{column}%
				\begin{column}{0.45\textwidth}%
					\begin{figure}%
						\centering%
						\begin{overprint}%
							\onslide<1>\includegraphics[width=0.85\textwidth]%
											  			{figures/results/2D/44426/phi.png}%
							\caption*{{\scriptsize%
								(Potential, \SI{5}{\pascal}, \SI{400}{\volt},\linebreak%
								 $U_{sb}=$\SI{-400}{\volt})}}%
								 \onslide<2>\includegraphics[width=0.85\textwidth]%
														{figures/results/2D/44426/i_dens.png}%
							\caption*{{\scriptsize%
								(Ionendichte, \SI{5}{\pascal},\linebreak%
								 \SI{400}{\volt}, $U_{sb}=$\SI{-400}{\volt})}}%
							\onslide<3>\includegraphics[width=0.85\textwidth]%
														{figures/results/2D/44426/i_velz.png}%
							\caption*{{\scriptsize%
								(axiale Ionengeschwindigkeit,\linebreak%
								 \SI{5}{\pascal}, \SI{400}{\volt}, %
								 $U_{sb}=$\SI{-400}{\volt})}}%
						\end{overprint}%
					\end{figure}%
				\end{column}%
			\end{columns}%
		\end{frame}%
		\nologoinbg
%
% Einfluss des Self Bias:
		\begin{frame}{Einfluss des Self Bias}
			\onslide<1->{%
				\begin{exampleblock}{}%
						Was für einen Einfluss hat der Self-Bias auf die Dynamik der O$^{-}$ %
						von der Oberfläche?
				\end{exampleblock}%
			}
			\only<2->{%
				\begin{figure}%
					\begin{subfigure}{0.45\textwidth}
						\centering%
						\includegraphics[width=0.75\textwidth]%
							  			{figures/results/2D/SFB/ni_dens_25830000.png}%
					\end{subfigure}
					\begin{subfigure}{0.45\textwidth}
						\centering%
						\includegraphics[width=0.75\textwidth]%
							  			{figures/results/2D/SFB/velni_z_25830000.png}%
					\end{subfigure}
					\caption*{{\scriptsize%
								(Ionendichte und axiale Geschwindigkeit einer Entladung %
								bei \SI{6}{\pascal}, \SI{400}{\volt} und $U_{sb}=$\SI{-200}{\volt})}}%
				\end{figure}%
			}
		\end{frame}
		\logoinbg
%
		\begin{frame}{Einfluss des Self-Bias}%
			\begin{columns}%
				\begin{column}{0.45\textwidth}%
					\only<1-3>{%
						\begin{block}{}%
							\onslide<1-3>{\stichpunkt{%
								hohe, axiale Geschwindigkeit der Ionen}}%
							\onslide<2-3>{\stichpunkt{%
								abbremsen in der Randschicht bzw. Bulk-Kante}}%
							\onslide<3>{\stichpunkt{%
								signifikanter Anteil der an der linken Elektrode erzeugten Teilchen erreicht andere Seite}}%
						\end{block}
					}
					\only<4-5>{%
						\begin{alertblock}{Erkenntnis}%
							Die negativen Ionen treffen auf die Anode mit hoher kinetischer Energie aufgrund der zusätzlichen Beschleunigung in der Randschicht durch den Self-Bias.
						\end{alertblock}
					}
					\only<6-7>{%
						\begin{alertblock}{Erkenntnis}%
							\onslide<6-7>{\stichpunkt{stoßbehaftete Schicht $\Rightarrow$ kalte Ionen im Bulk}}%
							\onslide<7>{\stichpunkt{Asymmetrie und Self-Bias notwendig zur Beschreibung der Effekte}}%
						\end{alertblock}
					}
				\end{column}%
				\begin{column}{0.45\textwidth}%
					\begin{figure}
						\begin{overprint}
							\centering
							\onslide<1-3>\includegraphics[width=0.8\textwidth]%
											{figures/results/2D/SFB/velni_z_25830000.png}%
							\caption*{{\scriptsize%
								(axiale Ionengeschwindigkeit,\linebreak%
									\SI{6}{\pascal}, \SI{400}{\volt} und $U_{sb}=$\SI{-200}{\volt})}}%
							\onslide<4>\includegraphics[height=0.55\textheight]%
											{figures/results/2D/SFB/ni_distz.png}%
							\caption*{{\scriptsize%
								(axiale Ionenenergieverteilung,\linebreak%
							\SI{6}{\pascal}, \SI{400}{\volt} und $U_{sb}=$\SI{-200}{\volt})}}%
							\onslide<5>\includegraphics[height=0.5\textheight]%
											{figures/results/2D/SFB/ni_cut.png}%
							\caption*{{\scriptsize%
								(Ionenenergieverteilung an der Anode,\linebreak%
							\SI{6}{\pascal}, \SI{400}{\volt})}}%
							\onslide<6-7>\includegraphics[height=0.55\textheight]%
											{figures/results/2D/SFB/ni_distz.png}%
							\caption*{{\scriptsize%
								(axiale Ionenenergieverteilung,\linebreak%
							\SI{6}{\pascal}, \SI{400}{\volt} und $U_{sb}=$\SI{-200}{\volt})}}%
						\end{overprint}%
					\end{figure}%
				\end{column}%
			\end{columns}%
		\end{frame}%
%
%
% Ausblick:
	\section{Ausblick}
%
		\begin{frame}{Fazit \& Ausblick}
			\begin{columns}
				\begin{column}{0.46\textwidth}
					\only<1>{\begin{alertblock}{Zusammenfassung}%
						\stichpunkt{Äquivalenz von axialen Profilen einer 2D-PIC Simulation und einem 1D-Model gezeigt}%
						\stichpunkt{Asymmetrie und Self-Bias genutzt um Dynamik negativer Ionen in CCRF-Entladungen zu studieren}%
					\end{alertblock}}
					\only<2>{\begin{figure}
						\centering
						\includegraphics[width=\textwidth]%
								{figures/results/2D/SFB/power_energy_cuts.png}%
						\caption*{{\scriptsize%
								(Simulationsergebnisse, O$^{-}$ Verteilung)}}%
					\end{figure}}
				\end{column}
				\begin{column}{0.46\textwidth}
					\only<1>{\begin{exampleblock}{Ausblick}%
						\stichpunkt{Verbesserung durch tatsächliche Stoßquerschnitte und Koeffizienten der Oberflächenprozesse}%
						\stichpunkt{2D-PIC für andere Modelle, wie zbsp. Sputtering nutzen}%
					\end{exampleblock}}
					\only<2>{\begin{figure}
						\centering
						\includegraphics[width=\textwidth]%
								{figures/neg_mg.png}%
						\caption*{{\scriptsize%
								(Experimentergebnisse, O$^{-}$-Verteilung)}}%
					\end{figure}}
				\end{column}
			\end{columns}
		\end{frame}
%
%
% Referenzen:
%
		\begin{frame}{Referenzen}
			\printbibliography
		\end{frame}
%
%
\end{document}
