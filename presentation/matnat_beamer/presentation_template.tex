% Dies ist eine Beispielpräsentation
% Tino Soll, November 2016
% 
\documentclass{beamer}

%%%%%%%%%%%%
% PRÄAMBEL %
%%%%%%%%%%%%

%%%%%%%%%%%%%%%%%%%%%%%%%%%%%%%%%%%%%%%%%%%%%%%%%%%%%%%%%%%%%%%
% Hier können Pakete mit \usepackage{...} eingebunden werden: %
%%%%%%%%%%%%%%%%%%%%%%%%%%%%%%%%%%%%%%%%%%%%%%%%%%%%%%%%%%%%%%%
%
% Deutsches Sprachpaket (u.a. für deutsche Datumsschreibweisen)
\usepackage[german]{babel}
% Paket zum Einbinden von Grafiken
\usepackage{graphicx}

% Auswahl der Vorlage "UniGreifswald"
\usetheme{UniGreifswald}

%%%%%%%%%%%%%%%%%%%%%%%%%%%%%%%%%%%%%%%%%%%%%%%%%%%%%%%%%%%%%%%
% Definitionen eigener Befehle \newcommand{...}{...}:         %
%%%%%%%%%%%%%%%%%%%%%%%%%%%%%%%%%%%%%%%%%%%%%%%%%%%%%%%%%%%%%%%



%%%%%%%%%%%%%%%%%%%%%%%%%%%%%%%%%%%%%%%%%%%%%%%%%%%%%%%%%%%%%%%%%%%%%%%%%%%%
% Annpassung der Fußzeile mit \renewcommand{\footlineinfo}{Eigener Text}:  %
%%%%%%%%%%%%%%%%%%%%%%%%%%%%%%%%%%%%%%%%%%%%%%%%%%%%%%%%%%%%%%%%%%%%%%%%%%%%
% Beispiel:
% \renewcommand{\footlineinfo}{\insertshortauthor (\insertshortinstitute) : \ \insertshorttitle}


%%%%%%%%%%%%%%%%%%%%%%%%%%%%%%%%%%%%%%%%%%%%%%%%%%%%%%%%%%%%%%%%%%%%%%%%%%%%%%%%%%%%%%%%%%
% Eckdaten der Präsentation: Titel, Autor, Institution, Vortragsort/Datum:               %
%%%%%%%%%%%%%%%%%%%%%%%%%%%%%%%%%%%%%%%%%%%%%%%%%%%%%%%%%%%%%%%%%%%%%%%%%%%%%%%%%%%%%%%%%%
% Die Angaben, die sich in den eckigen Klammern befinden sind jeweils Kurzformen des     %
% Titels, Autors, etc. und optional. Diese werden in der Fußzeile auf jeder Folie        %
% stehen, sofern \footlineinfo nicht modifiziert wurde.                                  %
% Wenn keine Kurzform angegeben wird, wird die "normale" Form genutzt.                   %
% Falls diese Aurgumente nicht gesetzt werden, sollten die eckigen Klammern              %
% fortgelassen werden (also nicht leer sein).                                            %
% Auf der Titelfolie werden stets die langen Formen aufgeführt.                          %
%%%%%%%%%%%%%%%%%%%%%%%%%%%%%%%%%%%%%%%%%%%%%%%%%%%%%%%%%%%%%%%%%%%%%%%%%%%%%%%%%%%%%%%%%%
% Titel der Präsentation
\title{Titel der Präsentation}
%\title[Titel (Kurzform)]{Titel der Präsentation}

% Untertitel der Präesentation (optinal)
\subtitle{Untertitel}
%
% Autor/Vortragender
\author{E. Vortragender}
%\author[Vortragender (Kurzform)]{E.~Vortragender}
%
% Institution:
% \institute{Universität Greifswald}
\institute[Uni Greifswald] % (optional)
{ 
	Institut für , \\
  	Ernst-Moritz-Arndt-Universität Greifswald
}
%
% Anlass des Vortrages / Konferenzort / Datum
\date[Konf./Dat.] % (optional)
{Konferenz/Konferenzort/Datum}
%\date{\today}


%%%%%%%%%%%%%%%%%%%%%%%%%%%%%%%%
%     BEGINN DES DOKUMENTS     %
%%%%%%%%%%%%%%%%%%%%%%%%%%%%%%%%
\begin{document}

% Erstellen der Titelfolie:
\begin{frame}
\maketitle
\end{frame}

% Folie mit Inhaltsverzeichnis:
\begin{frame}
\frametitle{Übersicht}
\tableofcontents
\end{frame}

\section{Hintergrundlogo}
\subsection{Logo ausschalten}

% Hintergrundlogo ausschalten
\nologoinbg

\begin{frame}{\insertsection}{\insertsubsection}
 Das Wasserzeichen-Logo lässt sich mit {\tt $\backslash$nologoinbg} für die folgenden Folien ausschalten. Der Befehl muss dazu VOR der gewünschten Folie im Quelltext platziert werden.
\bigskip
 
\end{frame}


\subsection{Logo einschalten}

% Hintergrundlogo einschalten
\logoinbg
\begin{frame}{\insertsection}{\insertsubsection}
Um das Hintergrundlogo wieder einzuschalten, muss der Befehl {\tt $\backslash$logoinbg }
im Quelltext VOR die entsprechende Folie gesetzt werden.
\end{frame}


\section{Aufzählungen}

\begin{frame}{\insertsection}
Hier zwei Beispiele, wie Aufzählungen in diesem Design umgesetzt wurden:

\bigskip

\hbox{
\begin{minipage}{0.5\textwidth}
\begin{enumerate}
\item Eins
\item Zwei
\begin{enumerate}%[a)] 
\item Drei
\item Vier
\begin{enumerate}
\item Fünf
\item Sechs
\end{enumerate}
\end{enumerate}
\end{enumerate}
\end{minipage}

\begin{minipage}{0.5\textwidth}
\begin{itemize}
\item Eins
\item Zwei
\begin{itemize}
\item Drei
\item Vier\begin{itemize}
\item Fünf
\item Sechs
\end{itemize}
\end{itemize}
\end{itemize}
\end{minipage}
}

\end{frame}

\end{document}