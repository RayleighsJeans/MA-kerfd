\chapter{Appendix}
%
% ************************************************************************** %
% PHYSICAL PROPERTIES
  \section{Physical Properties}
  \begin{longtable}{m{0.32\textwidth}m{0.32\textwidth}m{0.32\textwidth}}
    \toprule
    \bfseries quantity & \bfseries equation & \bfseries relevance \\%
    \toprule\midrule\endhead%
      Debye length &%
        $\begin{aligned}
          \lambda\ix{D,j}^2&=\frac{\varepsilon\ix{0}k\ix{B}T\ix{j}}{n\ix{j}e^2} \\
          \lambda\ix{D}^2&={\left(\lambda\ix{D,e}^{-2}+\lambda\ix{D,i}^{-2}\right)}^{-1}
        \end{aligned}$ &%
          distance around a charge, at which quasi-neutrality is satisfied, %
          $\lambda\ix{D}$ is the combined screening length from individual species \\ \midrule%
      plasma parameter &%
        $\begin{aligned}
          N\ix{D} = n\frac{4}{3}\pi\lambda\ix{D}^{3}
        \end{aligned}$ &%
        number of particles inside Debye sphere, if $N\ix{D} \gg 1$ an ionized gas %
        is considered a plasma (degree of ionization) \\ \midrule%
      plasma frequency &%
        $\begin{aligned}
          \omega\ix{p,j}^2=\frac{n\ix{j}e^2}{\varepsilon\ix{0}m\ix{j}}=%
          \frac{v\ix{th,j}}{\lambda\ix{D,j}}=\frac{1}{\tau\ix{j}}
        \end{aligned}$ &%
          upper limit for interaction with fields/forces or external excitations %
          inverse screening time \\ \midrule%
      thermal velocity &%
        $\begin{aligned}
          v\ix{th,j}^2=\frac{k\ix{B}T\ix{j}}{m\ix{j}}
        \end{aligned}$ &%
          mean velocity from kinetic theory of gases \\ \midrule%
      coulomb logarithm &%
        $\begin{aligned}
          &\ln\left(\Lambda\right) \\ \\
          &\Lambda=\frac{b\max}{b\min}= \\ \\
          &\lambda\ix{D}\cdot%
          \frac{4\pi\varepsilon\ix{0}\mu v\ix{th}^{2}}{e^{2}} 
        \end{aligned}$ &%
          dimensionless scale for transport processes inside discharge \newline
          fraction of probability for a cumulative $90^{\circ}$ scattering by many small %
          pertubation collisions and a single right angle scattering \\ \midrule%
      collision frequency &%
        $\begin{aligned}
          \nu\ix{j}=\frac{e^{4}n\ix{j}\ln\left(\Lambda\right)}%
          {8\sqrt{2m\ix{j}}\pi\varepsilon\ix{0}{\left(k\ix{B}T\ix{j}\right)}^{3/2}}
        \end{aligned}$ &%
          two body coulomb collision frequency inside species j \\ \midrule%
      particle distance \& \newline mean free path &%
        $\begin{aligned}
          &\overline{b}=\frac{\hbar}{m\ix{j}v\ix{th,j}} \\ \\
          &s\ix{mfp,j}=\frac{v\ix{th,j}}{\nu\ix{j,k}}
        \end{aligned}$ &%
          mean inter particle distance for species j \newline% 
          free flight between subsequent collisions of species j and k %
          with collision frequency $\nu\ix{j,k}$ \\ \midrule%
      speed of sound &%
        $\begin{aligned}
          c\ix{S}^{2}&=\frac{\gamma Zk\ix{B}T\ix{e}}{m\ix{i}} \\
          \gamma&=1+2/f=5/3
        \end{aligned}$ &%
        speed of longitudinal ion waves at electron pressure \newline%
        adiabatic coefficient with f, the kinetic degree of freedom\\ \midrule%
      Debye-Hückel potential &%
        $\begin{aligned}
          \Phi=\frac{Q}{4\pi\varepsilon|\vec{r}|}%
          \euler^{-\frac{|\vec{r}|}{\lambda\ix{D}}}
        \end{aligned}$ &%
        electrostatic potential of charge particle $Q$ at distance $|\vec{r}|$, \newline%
        equal to coulomb interaction with additional%
        shielding by charged particles \\ \midrule%
      drift velocity &
        $\begin{aligned}
          v\ix{d,j}=u\ix{j}=\frac{j\ix{j}}{n\ix{j}q}=\frac{m\sigma E}{\rho ef}
        \end{aligned}$ &%
        average velocity of a particle in a conductor with an electric field applied E, \newline%
        where $N$ is the number of free electrons per atom \\%
      electric mobility &
        $\begin{aligned}
          \mu\ix{j}=\frac{v\ix{d}}{E}
        \end{aligned}$ &%
        ability of charged particle of moving through an electric field \\%
    \midrule\bottomrule%
    \caption[Selection of physical properties of a low temperature ccrf discharge]{%
      Selection of physical properties of a low temperature ccrf discharge. The index $j$ denotes the %
      species, e.g.\@ electrons, ions. Used quantities can be found in the preface %
      in~\autoref{tabe:physicalconstants}.}\label{tabe:physicalquantities}
  \end{longtable}
% ************************************************************************** %
%   
		\clearpage
    \section[Energy Distributions from 2D PIC]%
            {Simulated Energy Distribution\\
            Functions from 2D PIC}\label{sec:appendix_results}
%
        \begin{center}
            \begin{figure}[!h]
                \centering
                \begin{subfigure}{0.49\textwidth}
									\includegraphics[height=0.3\textheight]%
                        {figures/results/2D/44332/e_dens.png}
                \end{subfigure}
                \begin{subfigure}{0.49\textwidth}
									\includegraphics[height=0.3\textheight]%
                        {figures/results/2D/44426/e_dens.png}
                \end{subfigure}
                %\caption[2D electron and ion density]{%
                %    Electron density from the two previously %
								%		described asymmetric 2D simulations.}
                %\label{fig:app_dens}
            %\end{figure}
						%\begin{figure}
								%\centering
                \begin{subfigure}{0.49\textwidth}
									\includegraphics[height=0.3\textheight]%
                        {figures/results/2D/44332/ni_dens.png}
                \end{subfigure}
                \begin{subfigure}{0.49\textwidth}
									\includegraphics[height=0.3\textheight]%
                        {figures/results/2D/44426/ni_dens.png}
                \end{subfigure}
								\newline
                %\caption[2D electron and ion density]{%
                %    Negative ion density from the two previously %
								%		described asymmetric 2D simulations.}
                %\label{fig:app_dens_ni}
            %\end{figure}
						%\begin{figure}[!b]
                %\centering
                \begin{subfigure}{0.49\textwidth}
									\includegraphics[height=0.22\textheight]%
                        {figures/results/2D/44332/e_distz.png}
                \end{subfigure}
                \begin{subfigure}{0.49\textwidth}
									\includegraphics[height=0.22\textheight]%
                        {figures/results/2D/44426/e_distz.png}
                \end{subfigure}
               	\caption[Electron and negative ion panel]{%
									\fett{Top}: electron density, \fett{Mid}: negative ion density, %
									\fett{Bottom}: axial compontent of electron EDF from a 2D simulation %
										of the two asymmetrical discharges discussed before-hand.}
                \label{fig:app_dens}
            \end{figure}
						\clearpage
            \begin{figure}
                \centering
                \begin{subfigure}{0.49\textwidth}
                    \includegraphics[width=1.0\textwidth]%
                        {figures/results/2D/44332/i_distz.png}
                \end{subfigure}
                \begin{subfigure}{0.49\textwidth}
                    \includegraphics[width=1.0\textwidth]%
                        {figures/results/2D/44426/i_distz.png}
                \end{subfigure}
                \caption[Axial ion EDF from 2D]{%
                    Axial compontent of ion EDF from a 2D simulation %
										of the two asymmetrical discharges discussed before-hand.}
                \label{fig:app_edf_i}
            \end{figure}
						\vfill
            \begin{figure}
                \centering
                \begin{subfigure}{0.49\textwidth}
                    \includegraphics[width=1.0\textwidth]%
                        {figures/results/2D/44332/ni_distz.png}
                \end{subfigure}
                \begin{subfigure}{0.49\textwidth}
                    \includegraphics[width=1.0\textwidth]%
                        {figures/results/2D/44426/ni_distz.png}
                \end{subfigure}
                \caption[Axial negative ion EDF from 2D]{%
                    Axial compontent of negative ion EDF from a 2D simulation %
										of the two asymmetrical discharges discussed before-hand.}
                \label{fig:app_edf_ni}
            \end{figure}
        \end{center}
%
		\clearpage
    \listoffigures % Prints the list of figures
	\listoftables % Prints the list of tables
