\chapter{Appendix}

% ************************************************************************** %
% PHYSICAL PROPERTIES
  \begin{longtable}{m{0.32\textwidth}m{0.32\textwidth}m{0.32\textwidth}}
  %\begin{longtable}{ccc}
    \toprule
    \bfseries quantity & \bfseries equation & \bfseries relevance \\%
    \toprule\midrule\endhead%
      Debye length &%
        \begin{align*}
          \lambda\ix{D,j}^2=\frac{\varepsilon\ix{0}k\ix{B}T\ix{j}}{n\ix{j}e^2} \\
          \lambda\ix{D}^2={\left(\lambda\ix{D,e}^{-2}+\lambda\ix{D,i}^{-2}\right)}^{-1}
        \end{align*} &%
          distance around a charge, at which quasi-neutrality is satisfied,%
          $\lambda\ix{D}$ is the combined screening length from individual species \\ \midrule%
      plasma parameter &%
        \begin{align*}
          N\ix{D} = n\frac{4}{3}\pi\lambda\ix{D}^{3}
        \end{align*} &%
        number of particles inside Debye sphere, if $N\ix{D} \gg 1$ an ionized gas %
        is considered a plasma (degree of ionization) \\ \midrule%
      plasma frequency &%
        \begin{align*}
          \omega\ix{p,j}^2=\frac{n\ix{j}e^2}{\varepsilon\ix{0}m\ix{j}}=%
          \frac{v\ix{th,j}}{\lambda\ix{D,j}}=\frac{1}{\tau\ix{j}}
        \end{align*}&%
          upper limit for interaction with fields/forces or external excitations%
          inverse screening time \\ \midrule%
      thermal velocity &%
        \begin{align*}
          v\ix{th,j}^2=\frac{k\ix{B}T\ix{j}}{m\ix{j}}
        \end{align*}&%
          mean velocity from kinetic theory of gases \\ \midrule%
      coulomb logarithm &%
        \begin{align*}
          \ln\left(\Lambda\right) \\
          \Lambda=\frac{b\max}{b\min}=\lambda\ix{D}\cdot%
          \frac{4\pi\varepsilon\ix{0}\mu v\ix{th}^{2}}{e^{2}} 
        \end{align*}&%
          dimensionless scale for transport processes inside discharge \newline
          fraction of probability for a cumulative $90^{\circ}$ scattering by many small %
          pertubation collisions and a single right angle scattering \\ \midrule%
      collision frequency &%
        \begin{align*}
          \nu\ix{j}=\frac{e^{4}n\ix{j}\ln\left(\Lambda\right)}%
          {8\sqrt{2m\ix{j}}\pi\varepsilon\ix{0}{\left(k\ix{B}T\ix{j}\right)}^{3/2}}
        \end{align*} &%
          two body coulomb collision frequency inside species j \\ \midrule%
      particle distance \& \newline mean free path &%
        \begin{align*}
          \overline{b}=\frac{\hbar}{m\ix{j}v\ix{th,j}} \\
          s\ix{mfp,j}=\frac{v\ix{th,j}}{\nu\ix{j,k}}
        \end{align*} &%
          mean inter particle distance for species j \newline% 
          free flight between subsequent collisions of species j and k %
          with collision frequency $\nu\ix{j,k}$ \\%
      \midrule\bottomrule%

      speed of sound &%
        \begin{align*}
          c\ix{S}^{2}=\frac{\gamma Zk\ix{B}T\ix{e}}{m\ix{i}} \\
          \gamma=1+2/f=5/3
         \end{align*} &%
        speed of longitudinal ion waves at electron pressure \newline%
        adiabatic coefficient with f, the kinetic degree of freedom\\ \midrule%
      Debye-Hückel potential &%
        \begin{align*}
          \Phi=\frac{Q}{4\pi\varepsilon|\vec{r}|}%
          \euler^{-\frac{|\vec{r}|}{\lambda\ix{D}}}
        \end{align*} &%
        electrostatic potential of charge particle $Q$ at distance $|\vec{r}|$ \newline%
        equal to coulomb interaction with additional%
        shielding by charged particles \\ \midrule%
      drift velocity &
        \begin{align*}
          v\ix{d,j}=u\ix{j}=\frac{j\ix{j}}{n\ix{j}q}=\frac{m\sigma E}{\rho ef}
        \end{align*} &%
        average velocity of a particle in a conductor with an electric field applied E, \newline%
        where $N$ is the number of free electrons per atom\\ \midrule
      electric mobility &
        \begin{align*}
          \mu\ix{j}=\frac{v\ix{d}}{E}
        \end{align*} &%
        ability of charged particle of moving through an electric field %
        --- with presence of a conductor \\%

    \midrule\bottomrule%
    \caption{%
      Selection of physical properties of a low temperature ccrf discharge. The index $j$ denotes the %
      species, e.g.\@ electrons, ions. Used quantities can be found in the preface %
      in~\autoref{tabe:physicalconstants}.}\label{tabe:physicalquantities}
  \end{longtable}
% ************************************************************************** %
