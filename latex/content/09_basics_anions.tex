%			
		\subsection{Anion Species}\label{sec:anionproduction}
%
			The main production channel of negative oxygen ions in ccrf disharges at low pressures and temperatures is the dissociative attachment reaction $(10)$. Here, an electron becomes attached to a molecule. The successive electronic excitation is of short duration and does not change the intra-molecular distance. Afterwards, theres a significant chance of transition to a dissociative state exists, which has a lower equilibrium energy at greater intra-nuclear distances. Hence, the dissociation of this molecule is rather likely.
%
			\begin{align}
				e^{-}+AB\rightarrow A+B^{-}%
				\label{equ:dissociative_attach}
			\end{align}
%
			Another possible creation channel is a three-body collision of non-dissociative character, whose cross sections is magnitudes smaller than the one of \autoref{equ:dissociative_attach}. Hence I will only consider dissociative attachment reactions $(10)$ for the anion production.\\
			Negative ion loss can happen through reactions $(11)$, $(13)$ and $(9)$. The latter is the only collision with a cross section larger than the creation via dissociative attachment (see~\autoref{fig:cross_sections}). For all relative energies, the neutralization has a probability of at least one magnitude larger than the other channels. Cross sections of direct $(11)$ and impact $(13)$ detachment are, depending on the energy, about one to two orders of scale smaller.\\
			In general, the produced negative ions are cold. The anion distribution reaches until the boundaries of the bulk, where processes with large cross sections at low energies become important~\cite{Bronold07b}. Those reactions would be ion-ion neutralization and associative detachment. Direct detachment, though being still present around $E<\unit[1]{eV}$, has an energy threshold and is not significant for this region. Furthermore, the probability of neutralization $(9)$ is proportional to the $O\ix{2}^{+}$-density. Bronold et al.\@ proposes, that the production and loss of $O^{-}$ is rather insensitive to voltage changes up to $\unit[300]{V}$. Furthermore, the most important range for incident energies will be $4$--$\unit[15]{eV}$, while the EEDF is rather voltage-independent.\\
			Considering the physics of a negative ion --- $O^{-}$ follows the same dynamic and kinetic behaviour as the electrons, but is easily confined by the plasma potential due to their much greater mass and, hence $\omega\ix{p,i-}<<\omega\ix{p,e}$ --- the main loss and production channels are most prominent in the bulk. Therefore, a low-pressure, low-temperature ccrf discharge has an electronegative core, in which the cold anions are captured, and areas where they are excluded. The presence of negative ions also has a great impact on the distribution functions of other plasma species. It is possible to form a quasi-neutral volume core, consisting only of ion species, and a peripheral electron-ion plasma in the discharge sheaths. At pressures $>\unit[30]{Pa}$ and large input powers, the value of electronegativity $\alpha$ leads to instabilities between ionization and electron attachment reactions. The electron density peaks, where as the corresponding temperatures drops. Because of the strong negative ion coupling, the $O^{-}$ density fluctuates as well.
