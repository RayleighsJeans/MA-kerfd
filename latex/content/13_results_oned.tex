%
\chapter{Validation of Simulation by 1d comparison}\label{sec:chapter_onedcomparison}
%
    Before we can investigate the processes in the two-dimensional PIC simulation that has been highlighted in the previous sections, one has to validate the quality of the model. For this purpose a well-established 1d3v Particle-in-Cell code (~\cite{Birdsall91,Matyash07oxIII,Matyash07PIC}~) is used to compare the results of the 2d3v simulation. One will also discuss the drawbacks and deviations of the one-dimensional code due to the many simplifications with respect to the experiment and two-dimensional expansion. The acquired experiences will create the foundation for the investigations of the next chapter, where the 2D-code is used to model important secondary emission processes.
%
    \section{1d3v PIC}
%
        The one-dimension simulation follows the same principles that have been discussed in~\autoref{sec:picsimulationmcc} and~\autoref{sec:pic_2d3v}. Densities, potential, electric field and all other macro-quantities are calculated at grid points based on the positions and velocities of pseudo-particles, which represent a large number of physical particles. They are pushed using a weighting scheme to interpolate the electric field at the fixed grid points to the continuously defined particle positions. The grid is equidistant with distance $\Delta z\ix{0}=\lambda\ix{D,e}/2$. The 1D code resolves only one spatial dimension, but $\vec{v}=(v\ix{r},v_{\vartheta},v\ix{z})$ completes the four-dimensional phase-space. As for the two-dimensional simulation model, we are only interested in oxygen plasmas, and the same collisions and cross-section data are used. They are discussed in~\autoref{sec:negiondynamics} and shown in~\autoref{fig:cross_sections}.\\
        The simulation model resembles a parallel plate rf discharge. The electrodes are placed at both ends of the domain, e.g.\@ $x=0$ and $x=N\ix{z}\cdot\Delta z\ix{0}$, and driven at a frequency of $\SI{13.56}{\mega\hertz}$ and voltages between 100--$\SI{1000}{\volt}\,$. Both cathode and anode are assumed to be totally absorbing. Charged particles are randomly distributed across the whole domain with respect to the initial electron density $n\ix{e,0}$. The neutral gas is treated with $n\ix{n}=const.$ as an inexhaustible reservoir with fixed temperature $T=\SI{300}{\kelvin}$, and the pressure was chosen between 2--$\SI{30}{\pascal}$.\\
%        
%       Like it was discussed by Bronold and Matyash et al. in~\cite{Bronold07b}, the key argument for a one-dimensional simulation is the large electrode diameter in comparison to the electrode gap. Here it is said that the plasma properties are not affected, or at least the influence is considered negligible, by the boundaries of the electrodes. Along the axial centre of the discharge the plasma `does not see' the edge of the electrodes and therefore no asymmetry effects should take place~\cite{Matthias15}. A radial dependence of the plasma parameters close to the symmetry axis can be neglected~\cite{Matyash07PIC}.\\
        
    