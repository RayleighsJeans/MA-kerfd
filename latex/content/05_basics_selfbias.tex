%
        \section{Sheath Effects in RF Discharges}\label{sec:selfbias}
%   
            In my thesis I am interested in asymmetric, capacitively coupled radio frequency plasmas. In the following sections I want to highlight the most important aspects of ccrf discharges for my work. A more in-depth approach can be found in, e.g.\@~\cite{Piel10}.\\
            The rf plasmas studied here are driven at a frequency of $\SI{13.56}{\mega\hertz}$. The modulation of the potential at the electrode within one rf cycle leads to time-dependent plasma properties. Because of $\omega\ix{p,i+}<\omega\ix{p,i-}<\omega\ix{rf}\ll\omega\ix{p,e}$ all plasma species interact with the rf voltage differently. The electrons follow the applied field instantaneously, where as the ions can be considered stationary. The modulation of the electron density to the rf induces also sheath oscillations of the electrons with the same frequency. A scheme of the plasma pre-/sheath can be seen in~\autoref{fig:sheath_piel}.\\
            Usually, ccrf plasmas have asymmetric characteristics. The asymmetry is created by the different areas of driven electrode and grounded walls. For example, let us assume that only the cathode is driven with rf, and that anode and plasma chamber are grounded resulting in a much larger area of the grounded part compared with the driven electrode area. The sheath then develops like explained above.\\
            If the cathode is capacitively coupled to the driver, there can not be any net current over one rf cycle. The accumulated charge on the electrode can not be flushed and the \emph{self bias} establishes to satisfy continuity $j\ix{e}=j\ix{i}$. The capacitance can not be inverted over the course of one rf cycle. The electron currents are then equal on both electrodes, therefore shifting the minimum plasma potential to ground and the maximum to the excitation. The dc \emph{self bias} part $U\ix{sb}$ and the mean plasma potential $\overline{\Phi\ix{p}}$ becomes:
%			
			\begin{figure}[!t]
				\centering%
				\includegraphics[width=1.0\textwidth]{figures/selfbiasvoltage.png}
				\caption[Schematics of voltage and plasma potential for different asymmetry configurations]{%
					Schematics of the voltage $U(t)$ and plasma potential $\Phi(t)$ %
					for a directly and capacitively coupled rf discharge. Different cases of %
					symmetry are shown: enlarged driven electron, grounded %
					electrode and a symmetric discharge.~\cite{Piel10}}
				\label{fig:circuitselfbias_2}
			\end{figure}
%
			\begin{align}
				\overline{\Phi\ix{p}}=\frac{1}{2}(U\ix{sb}+U\ix{rf})%
%				    \nonumber\\[0.0cm]
                    \quad\text{~and~}\quad%
    				U\ix{sb}=\frac{C\ix{1}-C\ix{2}}{C\ix{1}+C\ix{2}}U\ix{rf}%
					\,,%
					\label{eq:selfbiaszwei} 
			\end{align}
%
            where $C\ix{1,2}=\varepsilon\ix{p}\varepsilon\ix{0}\frac{A}{b}$ are the capacities of the corresponding plasma sheath. Here, $A$ and $b$ are cross section and thickness of the space charge volume, and $\varepsilon\ix{p}$ denotes the permeability of the working gas. Hence the value of the self bias becomes a function of discharge geometry, working gas and plasma sheath. For example, larger ratios of asymmetry, e.g.\@ $C\ix{1}\gg C\ix{2}$ lead to large values of $U\ix{sb}$. A common approximation for the self bias voltage is roughly half of the driven electrodes peak-to-peak voltage: $|U\ix{sb}|\approx U\ix{rf}/2$.\\
            Figure~\ref{fig:circuitselfbias_2} shows a scheme for different kinds of asymmetric setups, e.g\@ larger electrode or grounded wall sizes. The corresponding plasma potential and self bias voltage for a capacitive coupling are displayed below. One can see that the self bias heavily depends on the kind of asymmetry in the ccrf discharge.\\
			Both sheaths of the electrodes collapse completely during a full cycle of $U(t)$. At this moment, no potential barrier or space charge is hindering the particles to hit the electrodes. Electrons and ions can impinge on the surface and force the plasma potential $\Phi\ix{P}$ to level out with the walls. This short circuit between plasma and sheath occurs when $\Phi\ix{P}$ becomes negative with regard to the excitation:
%
			\begin{align}
				\Phi_{\text{p},\max}=\overline{\Phi\ix{p}}+\Phi\ix{rf}\geq U\ix{sb}+U\ix{rf}\,,%
%					\nonumber\\[0.0cm]
					\quad\quad%
					\Phi_{\text{p},\min}=\overline{\Phi\ix{p}}-\Phi\ix{rf}\geq0%
					\label{equ:selfbias_unequal}
			\end{align}
%
			If there is no special coupling between electrode and electrical driver, the equality in~\autoref{equ:selfbias_unequal} is true.

%
%           In the case of an asymmetric configuration with different electrode sizes, the potential inside the sheath can change drastically. This part of a plasma can have a huge impact on the behaviour of the discharge, e.g. an additional heating and energy dissipation into the plasma volume. Radio frequency plasmas are characterised by their transport process inside the sheath and heating mechanisms of charged species. A more in-depth discussion can be found in~\autoref{sec:heating}. The properties of the sheath are very important for plasma-assisted industrial applications, such as etching/sputtering and thin-film deposition.\\
%   	    An additional capacitance can be placed between electrode and generator. Such capacitively coupled rf discharges are characterised by a dc voltage offset on the electrode/s. This is called \emph{self bias} (see~\autoref{sec:selfbias}). The capacitor does not allow accumulated charges to flush to ground or the electric circuit. Furthermore, an additional current between plasma sheath and volume accommodates as a result of the different time scales of particle movement. This is called dielectric displacement current.\\
%
%            The time-dependent plasma potential $\Phi\ix{p}(t)$ and voltage across the discharge $U(t)$ are:
%
%			\begin{align}
%				U\left(t\right)=U\ix{sb}+U\ix{rf}\sin\left(\omega t\right)%
%					\nonumber\\[0.0cm]
%					\quad\text{~and~}\quad%
%					\Phi\ix{p}\left(t\right)=\overline{\Phi\ix{p}}+%
%					\Phi\ix{rf}\sin\left(\omega t\right)\,.%
%					\label{equ:selfbias_1}
%			\end{align}
% 
%                
%			\begin{wrapfigure}[13]{r}{0.4\textwidth}
%				\centering%
%				\vspace*{-0.5cm}%
%				\includegraphics[width=0.17\textwidth]{figures/circuit_selfbias_piel.png}
%				\caption{%
%					Replacement circuit of an asymmetrically driven ccrf %
%					discharge.~\cite{Piel10} A diode represents the directed electron %
%					current from the sheaths $j=1,2$.}\label{fig:replacementcurrent}
%			\end{wrapfigure}
%
%           A step towards the characterisation of asymmetric ccrf discharges is the development of a replacement circuit, see~\autoref{fig:replacementcurrent}. Thus, one can define a specific impedance for a rf discharge of excitation frequency $\omega$. The value of $\varepsilon\ix{p}$ resembles the permeability of the working gas between the driven and/or grounded electrode~\cite{Piel10}. In addition, this volume has the capacity $C\ix{p}$ --- the capacity of a cubicle with a cross section $A$, thickness $b$ and electron-neutral collision frequency $\nu\ix{e,n}$ calculates like~\autoref{equ:capacityandepsilon}.
%
%				\begin{align}
%					\varepsilon\ix{p}=&1-\frac{\omega\ix{p,e}^2}{\omega\left(\omega-\imag\nu\ix{e,n}\right)}\,,%
%						\quad\quad%
%						C\ix{p}=\varepsilon\ix{p}C\ix{0}=%
%						\varepsilon\ix{p}\varepsilon\ix{0}\frac{A}{b}%
%						\label{equ:capacityandepsilon}\\[0.2cm]
%					&Z\ix{p}={\left(\imag\omega C\ix{p}+ \frac{1}{\frac{1}{\omega\ix{p,e}^2C\ix{0}}%
%							{\left(\nu\ix{e,n}+\imag\omega\right)}}\right)}^{-1}%
%					\label{equ:bulkimpedanz}
%				\end{align}
%		
%				The~\autoref{equ:bulkimpedanz} represents the full electrical impedance, consisting of the inverse sum of real and imaginary resistance, as well as the capacity of the neutral gas volume. Here, $\imag\omega/(\omega\ix{p,e}^{2}C\ix{0})$ characterizes the electrons inertia in regard to an external excitation $\omega$. The real part $\nu\ix{e,n}/(\omega\ix{p,e}^{2}C\ix{0})$ denotes the resistance by neutral particle collisions. For high excitation frequencies, e.g.\@ $\SI{13.56}{\mega\hertz}$ the bulk impedance can be neglected (see~\autoref{equ:bulkimpedanz}~\cite{Kay85}). Both sheath capacities of anode and cathode take the dominant part.
%
%				If the excitation frequency $\omega$ is small compared to other time scales, e.g\@ electron and ion plasma frequencies, the electron current from the sheath becomes bigger than the displacement current. Hence the electron current onto the driven electrode decreases by a Maxwell factor --- this is a function of the there-on applied voltage --- compared to the corresponding ion current. Therefore the sheaths impedance is bigger than those of the floating walls. Together with~\autoref{equ:selfbias_1} and~\autoref{equ:inequality} the plasma potential $\Phi\ix{p}$ vanishes and only the currents onto the driven electrode have to be equal. With $\mathbf{I}\ix{0}$ the zeroth order modified \emph{Bessel function} the following equation yields the self bias voltage:
%      
%				\begin{align}
%					U\ix{sb}=\frac{k\ix{B}T\ix{e}}{e}\ln%
%						\left[\mathbf{I}\ix{0}\left(\frac{eU\ix{rf}}{k\ix{B}T\ix{e}}\right)\right]\,.%
%					\label{equ:selfbias_3}
%				\end{align}