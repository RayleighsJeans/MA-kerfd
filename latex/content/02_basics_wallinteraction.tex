%		
		\subsection{Plasma-Wall Interaction}\label{sec:sheathphysics}
%
			In the discharges bulk, neutral gas particles are excited by electron collisions and radiating visible light. However, areas around, e.g.\@ floating metal surfaces, probes and grounded walls are darker than the bulk. This is due to the low electron density and kinetic energy in this \emph{plasma sheath}. Though areas with vanishing electron numbers can glow because of high collision efficiencies and/or frequencies.\\
			Electrons, in general, are of a much higher mobility $\mu\ix{e}$ and thermal velocity $v\ix{th,e}$. Hence they impinge onto walls and surfaces more often than other species, leading to a --- in this case we consider an electronegative oxygen discharge, where the following can be assumed true --- negative charge and potential.
%
			\subsubsection{Child-Langmuir Law}\label{sec:langmuirlaw}
%
				For an asymmetric ccrf discharge, dc \emph{self bias} and displacement current are important parts of the electric system. Hence, the \emph{Child-Langmuir Law} as a function of those properties can be written. The rf component of the excitation is neglected.\newpage
				A greatly negative charged wall at $x=0$ shall be a barrier for electrons of thermal velocity, e.g.\@ $|\Phi(0)-\Phi(d)|\ll k\ix{B}T\ix{e}/e\,$. The thickness of the sheath shall be considered $d$. In an one-dimensional approach~\cite{Piel10}, the electron density $n\ix{e}(x)$ can be written with a \emph{Boltzmann} distribution function $f\ix{B}(\Phi)$:
%
				\begin{align}
					n\ix{e}(x)=n\ix{e}(d)\cdot f\ix{B}(\Phi)=%
					n\ix{e}(d)\cdot%
					\exp\left(\frac{e{\left(\Phi(x)-%
							\Phi(d)\right)}}{k\ix{B}T\ix{e}}\right)\quad.%
					\label{equ:electrondens}
				\end{align}
%
				This means that the electron density decreases exponentially towards the negatively charged wall. It can be assumed that the sheath thickness $d\ll s\ix{mfp,i}$ the mean free path of the ions inside the plasma bulk. Hence, ions enter the sheath collisionless.\\
				At the boundary between bulk and pre-sheath, the walls potential vanishes because of the plasmas shielding capabilities. Here, the ions are at $v\ix{i,0}$ speeds, therefore their density becomes:
%
				\begin{align}
					n\ix{i}(x)=n\ix{i}(d){\left(1-\frac{2e\Phi(x)%
							}{m\ix{i}v\ix{i,0}^{2}}\right)}^{-1/2}%
					\label{equ:iondens}
				\end{align}
%
				Futhermore, one can assume that the kinetic energy at this point is smaller than the potential energy for the acceleration inside the pre-sheath, e.g.\@ $m\ix{i}v\ix{i,0}^{2}\ll |e\Phi(x)|$. Using \emph{Poisson's } equation, and taking into account the ion-sheath interaction,~\autoref{equ:phibypoisson} gives an equation for $\Phi(x)$:
%
				\begin{align}
					\Delta\Phi\cong-\frac{en\ix{i}{\left(-d\right)}%
							}{\varepsilon\ix{0}}{\left(-\frac{2e\Phi{%
							\left(x\right)}}{m\ix{i}v\ix{i,0}^2}\right)}^{-\frac{1}{2}}%
					\label{equ:phibypoisson}
				\end{align}
%
				Solving this, and using the unpertubated ion current $j\ix{i}=n\ix{i}(d)ev\ix{i,0}$, one yields the result by \emph{Langmuir}.
%
				\begin{align}
					\Phi{\left(x\right)}={\left({\left(\frac{3}{4}{\left(x+d%
							\right)}\right)}^4{\left(\frac{j\ix{i}}{\varepsilon\ix{0}%
							}\right)}^2\frac{m\ix{i}}{2e}\right)}^{\frac{1}{3}}%
					\label{equ:langmuirpot}
				\end{align}
%
				Again, solving~\autoref{equ:langmuirpot} for the current $j\ix{i}$ yields the \emph{Child-Langmuir Law} (see~\autoref{equ:childlangmuirlaw}).\\
				This equation defines the ion current as a function of the unperturbed plasma bulk. In other words, the sheath changes its thickness in dependency of those certain discharge parameters, always satisfying the ion current defined by the \emph{Child-Langmuir Law}.
%
				\begin{align}
					j\ix{i}=\frac{4}{9}\varepsilon\ix{0}{\left(\frac{2e{%
							\left(\Phi{\left(-d\right)}-\Phi{\left(0\right)}%
							\right)}^3}{m\ix{i}d^2}\right)}^{\frac{1}{2}}%
					\label{equ:childlangmuirlaw}
				\end{align}
%
			\subsubsection{Surface Effects and Secondary Ion Emission}\label{sec:surfaceeffects}
%
				Although plasma sheath physics are influenced by bulk properties, such as temperatures and densities, the space charge area itself is mainly characterized by processes of the wall. Hence an important aspect is the absorption and re-emission of ions and electrons.	Those species than have unique features, like e.g.\@ high velocities. Let's assume the associated metal wall behind the sheath ideally absorbs all impinging, charged particles, which recombine immediately near the surface.\\
%
				\paragraph{Secondary Electron Emission}
				The discharges electrons are much faster and mobile than the other species, leading to the afore-mentioned negative charging and potential drop towards the wall. This accelerates the ion species up to \emph{Bohm velocity} --- see~\autoref{sec:bohmcriteria} for a more in-depth discussion of the \emph{Bohm criteria} and the ions sheath physics.\\
				Continuity and charge conservation must be satisfied, hence the fluxes $J\ix{j}$ of the species $j$ must be equal at the sheath edge --- into and out of the sheath:
%
				\begin{align}
					J\ix{e}=J\ix{i}%
					\label{equ:sheathequi}
				\end{align}
%
				The mentioned potential drop $\Delta\Phi$ from~\autoref{sec:langmuirlaw}, beside accelerating positive ions, reflects negative charges slower than $v\le \sqrt{2e\Delta\Phi/ m}\,$. Similar to~\autoref{equ:electrondens}, the electron current towards the wall can be written. Here the first moment of the electron velocity $\bra v\kett$ is used, calculated in~\autoref{equ:firstmoment} with the electron energy distribution function (EEDF) $f\ix{e}(v)$:
%
				\begin{align}
					j\ix{e}=&-\frac{e}{4}n\ix{e}\bra v\kett\exp\left(-\frac{e\Delta\Phi}{k\ix{B}T\ix{e}}\right)%
					\label{equ:electroncurrent}\\[0.0cm]
					\bra v\kett=&\int_{\mathbb{R}}v\cdot f\ix{e}\left(v\right)\diff v%
					\label{equ:firstmoment}
				\end{align}
%
				Impinging ions are neutralized before impact by particles from the electron gas in the wall. Like before, any produced neutrals are reflected and exit the sheath collisionless --- the mean free path is larger than the sheath thickness. Hence ionization is a process almost exclusively happening in the bulk or the pre-sheath.\\
				Assuming a fast electron impacts on the wall, there is a chance of colliding with and liberating a second electron from the target. Here, the \emph{secondary electron emission} coefficient is noted as $\gamma$: an impinging electron emits $\gamma$-many electrons from the metal. This \emph{SEE} reduces the $\Delta\Phi$ because of an addition charge current from the wall towards the sheath edge, therefore altering the continuity condition $j\ix{i}=j\ix{e}$. A new \emph{effective potential drop} $\Delta\Phi\ix{eff}$ can be written in~\autoref{equ:effectivepotentialdrop}. According to~\cite{Duras11} there is a critical value $\gamma\ix{c}$ from which on the wall potential is unstable, leading to shifting sheath edges --- the sheath edge oscillates with the rf signal anyway --- and strong currents from the wall.
%
				\begin{align}
					\Delta\Phi\ix{eff}=-\frac{k\ix{B}T\ix{e}}{e}\cdot\ln\left(\left(1-%
							\gamma\right)\sqrt{\frac{m\ix{i}}{2\pi m\ix{e}}}\,\right)%
					\label{equ:effectivepotentialdrop}
				\end{align}
%
				\paragraph{Secondary Ion Emission}	
				Research prior to this thesis~\cite{Kullig12} indicates that ions are produced near the surface of a metal electrode and heavily accelerated in the plasma sheath. In theory, secondary emission by surface ionization --- in analogy to the surface neutralization --- occurs with incident atoms of thermal energy. Hence one assumes a positively biased wall at high temperatures as the target. It's valence level is therefore broadened, giving an atom $A$ the chance the deposit an electron at the metal. After equilibrating thermally, a positive ion is emited by chance. This statistical process can be described by a thermodynamic equation (see~\autoref{equ:sahalangmuirpos}) yielding the ionization coefficient of $A$. In~\autoref{equ:sahalangmuirpos} a modified approach for the \emph{Saha-Langmuir equation} on the degree of ionization in gases can be found. Here, the surface's temperature $T$ and average work function $\overline{\Phi}_{+}$ are needed. Additionally, the ionization energy $I(A)$ --- or impact energy ---, the particle number currents of both species $j$ and $j^{+}$, corresponding statistical weights $w$, $w^{+}$ and reflection coefficients at the intrinsic potential barrier $r$/$r^{r}$ are used.
%
				\begin{align}
					A\leftrightharpoons A^{+}+&\,e^{-}%
					\nonumber\\[0.0cm]
					\alpha^{+}(A^{+})=\frac{j^{+}}{j}=\frac{(1-r^{+})\,w^{+}}{(1-r)\,w}\cdot&\exp\left(%
					\frac{\overline{\Phi}_{+}+e\sqrt{eV\ix{ext}}-I(A)}{k\ix{B}T}\right)%
					\label{equ:sahalangmuirpos}
				\end{align}
%
				At high temperatures of, e.g.\@ $\unit[1000]{K}$ and externally applied potentials $V\ix{ext}<\unit[1]{kV}$, the reflection associated \emph{Schottky term} $e\sqrt{eV\ix{ext}}$ and the corresponding coefficients $r$/$r^{+}$ can be neglected --- it appears to be just half of the thermal energy at room temperature. Though a theoretical approach is possible, there have not been accurate studies of such coefficients for temperatures around $\unit[300]{K}$.\\
				In addition to SIE of positive ions, a case for negative ions can be easily derived with small changes to~\autoref{equ:sahalangmuirpos}: a negatively biased electrode is assumed and the average work function yields a different sign. The electron affinity of the incident particle $B$ is noted as $A(B)$.
%
				\begin{align}
					B+\,&e^{-}\leftrightharpoons B^{-}%
					\nonumber\\[0.2cm]
					\alpha^{-}(B^{-})=\frac{(1-r^{-})%
						\,w^{-}}{(1-r)\,w}\cdot&\exp\left(%
						\frac{-\overline{\Phi}_{-}+e\sqrt{eV\ix{ext}}+A(B)}{k\ix{B}T}\right)%
					\label{equ:sahalangmuirneg}
				\end{align}
%
				Applying the former assumptions to both equations of positive and negative ions, inserting a homogeneous work function $\Phi=\overline{\Phi}_{-}=\overline{\Phi}\ix{+}$ for the used substrate yields the originally derived \emph{Saha-Langmuir equations}.
%				
				\begin{align}
					\alpha^{+}(A^{+})=\frac{w^{+}}{w}\,\exp\left(\frac{%
					\Phi-I(A)}{k\ix{B}T}\right)\,,%
						\quad\quad%
					\alpha^{-}(B^{-})=\frac{w^{-}}{w}\,\exp\left(\frac{%
						-\Phi+A(B)}{k\ix{B}T}\right)%
						\label{equ:sahalangmuirshort}
				\end{align}
%
				Though only considering atomic particle beams onto the wall until this point, forms similar to~\autoref{equ:sahalangmuirshort} can be derived for molecular surface interactions~\cite{Kawano83}. In case of the earlier discussed ccrf discharges, arguments like high temperatures can not be applied, hence the need for measured reflection coefficients.\\
				Works of, e.g.\@~\cite{Ustaze97} and~\cite{Los90} investigated ion beam scattering, electron loss and transport in plasma sheath environments for metal walls, especially $\text{MgO}(100)$ surfaces. There Ustaze et al.\ studied incident oxygen gas particles --- ions and neutrals ---  on magnesium oxyde surfaces. Impinging atoms became negatively charged ions, picking up electrons from the $\text{MgO}$ of the wall. This interaction, though requiring a minimum ionization and liberation energy for the electron, is most effective at low energies $<\unit[1]{eV}\,$. This is due to a maximum of residence time at the target for an incoming atom. Hence it can be considered a non-resonant charge transfer process at the anion site. For an more in-depth discussion of both electron loss and capture for anion transport processes at walls one should consider~\cite{Kawano83}.\\
				Here, due to the lack of theoretical and experimental data on ion surface production, an ion beam onto the wall will result in a anion current in opposing direction of
%
				\begin{align}
					j_{-}=\eta\,j\ix{+}\,\,,%
					\label{equ:negative_efficiency}
				\end{align}
%
				with a corresponding efficiency of an incident positive particle $\eta\,$. One should keep in mind the reaction of the plasma sheath and potential inside to an additional anion current from the wall. Also, the same stability criteria apply for $\eta$ as they do for the electron emission coefficient $\gamma$. In case of SIE beyond a critical value $\eta\ix{c}$, a second plasma sheath may develop, enclosing the bulk and an inner sheath and reducing transport in-between.
