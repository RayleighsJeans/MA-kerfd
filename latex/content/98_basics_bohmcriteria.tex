%			In summary, the ion dynamic discussed before is spatially restricted to the sheath and pre-sheath, where electron depletion occurs or an external, negative potential is applied.\\
%               In an one-dimensional model, a negatively charged wall at $x=0$ creates a potential barrier for electrons of thermal velocity, e.g. $|\Phi(0)-\Phi(d)|\ll k\ix{B}T\ix{e}/e\,$. The thickness of the space-charge sheath here is $d$ and the sheath-boundary therefore at $x=-d$ (see~\autoref{fig:sheath_piel}). It is usually the size of a few Debye-length $\lambda\ix{D}$.
%				Solving again for $j\ix{i}$ yields the \emph{Child-Langmuir Law} (see~\autoref{equ:childlangmuirlaw}). This equation defines the ion current as a function of the unperturbed plasma bulk. In other words, the sheath changes its thickness in dependency of the discharge parameters, always satisfying the ion current defined by the Child-Langmuir Law:
%
%				\begin{align}
%					j\ix{i}=\frac{4}{9}&\varepsilon\ix{0}{\left(\frac{2e{%
%						\left(\Phi{\left(-d\right)}-\Phi{\left(0\right)}%
%						\right)}^3}{m\ix{i}d^2}\right)}^{\frac{1}{2}}%
%				    	\label{equ:childlangmuirlaw}
%				\end{align}
%				
%				The \emph{Child-Langmuir Law} yields an expression for the relation between potential and current on a wall in a space-charge limited plasma region, e.g\@ the sheath. It characterises transport processes between a floating wall and the plasma bulk. It is important to understand the oscillation of the sheath in rf plasmas or secondary emission processes at walls.
%
%		\subsection{Plasma sheath}\label{sec:bohmcriteria}
%			
%			\begin{align} 
%				f{\left(\Phi\right)}=%
%					\Delta_{\vec{r}}\Phi=\frac{\diff^{2}\Phi}{\diff\vec{r}^{2}}=%
%					\frac{\rho}{\varepsilon\ix{0}}%
%				    \label{equ:pseudo}%\\[0.2cm]%
%			\end{align}%\vspace*{-18pt}
%			
%			For a more detailed of the plasma sheath the equations of motion for electrons and ions have to be solved, including the electric field from \emph{Poisson's} equation from~\autoref{equ:ionandelectrondens}. Because of the expression for $\Phi(x)$ in~\autoref{equ:langmuirpot}, which yields $\Delta\Phi\le0$ and therefore $\nabla E \le0$, we know that the ions acceleration increases towards the wall. Therefore the potential derivative of the function $f(\Phi)$ at the sheath-boundary, e.g\@ $\Phi=0$, is analogous to the gradient of the electric field in Maxwell's equation	