%
\chapter{Physical Properties of Low Temperature RF Plasmas}\label{sec:chapter_ccrfbasics}
%
	In this first chapter I will provide the necessary physical background for this work about the numerical simulation of low temperature, capacitively coupled radio frequency plasmas. Here I will firstly highlight the fundamental physics of a plasma and the interaction with walls. Upon that I will show the important features of asymmetric and capacitively coupled radio frequency discharges. Next the PIC simulation method with Monte-Carlo-Collision algorithms will be introduced. At last the necessary plasma chemistry and selected reaction set of oxygen will be presented.\\
	An important aspect of oxygen plasmas is the existence of negative ions in addition to electrons. Therefore, a negatively charged species with large mass exists in addition to the electrons. The ratio between anion and electron density is defined as electronegativity $\eta=n\ix{i-}/n\ix{e}$. This can range between 0.03--4~\cite{Kullig12} for the oxygen plasmas considered in this thesis. One key question to be addressed is the influence of negative ions on the physics of the plasma.
%
	\section{Plasma Physics}\label{sec:sheathphysics}
%
		The experiment studied in this work is a capacitively coupled radio frequency (ccrf) discharge with a low temperature plasma, operated at low pressures. A plasma is a globally quasi-neutral gas, consisting of freely moving charges --- e.g.\@ electrons, positive and negative ions --- with additional neutral gas particles. It is characterised by the collective behaviour of the charged species. The ratio of ion particle density ($n\ix{i}$) and of the sum of neutral ($n\ix{n}$) and ion densities defines the degree of ionisation. This is rather low for ccrf discharges, typically below 1\%.\\
		Charge separation of electrons and ions, and therefore the violation of the quasi-neutrality condition $n\ix{e}\,=\,n\ix{i}$ (electron density equals ion density), is only possible for distances below the \emph{Debye length} $\lambda\ix{D}$. Regions where the quasi-neutrality condition is satisfied are called plasma bulk. The definition of the degree of ionisation $\alpha$ and the Debye length are:
%
        \begin{align}
            \alpha=\frac{n\ix{i}}{n\ix{n}+n\ix{i}}\,,%
            \quad\quad%
            \lambda\ix{D}^2&={\left(\lambda\ix{D,e}^{-2}+%
                \lambda\ix{D,i}^{-2}\right)}^{-1}\,,%
            \quad\quad%
            \lambda\ix{D,j}^2=\frac{\varepsilon\ix{0}k\ix{B}T\ix{j}}%
                {n\ix{j}e^2}\,.%
            \label{equ:degreeionizdebye}
        \end{align}
%		
        where $\lambda\ix{D,j}$ are the individual Debye lengths of the species $j$. This is also the characteristic scale beyond which the electric potential of a charge is fully shielded.\\
        The creation of a plasma is accomplished by two parallel metal electrodes, where on at least one an ac or dc signal is applied. In this thesis I will consider alternating currents at radio frequency, $\SI{13.56}{\mega\hertz}$ with an amplitude between $100$--$\unit[1000]{V}$.\\