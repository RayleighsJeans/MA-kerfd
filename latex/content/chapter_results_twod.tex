%
\chapter{Simulation of capacitively coupled rf discharges}
%
  \section{Experimental setup}
%
	Giving the necessary background in~\autoref{sec:surfaceeffects} and providing the crucially important comparison of the previous~\autoref{sec:onedcomparison}, the afore-mentioned effects of highly energetic negative oxygen ions can now be further investigated.\\
	Here, the referenced experiment was first established and used by Kuellig et al.~\cite{Kullig12} and consists of a cylindrical setup, filled with oxygen at low pressures and gas flow rates. The stainless steel vacuum chamber had a diameter and height of $\unit[40]{cm}$ respectively and was filled with the oxygen (O$\ix{2}$) process gas at $\unit[5]{sccm}$. The discharge configuration consisted of an electrode with $\unit[10]{cm}$ of diameter and a rf generator, operating at a frequency of $\unit[13,56]{MHz}$ and power outputs between $5$ and $\unit[150]{W}$, leading to applied voltages in the range of $100$--$\unit[1500]{V}$. Shielding and discharge enclosure/chamber walls were grounded, therefore yielding a large area ratio between driven and grounded electrode and establishing a heavily asymmetric plasma. In addition, the powered electrode was coupled capacitively with the external generator, emphasizing the effect of the, in~\autoref{sec:selfbias} introduced earlier \emph{self bias voltage}. The value of $U\ix{sb}$ ranged, dependent on the power output, from $-100$ up to $\unit[500]{V}$.  \\
%	
	Standard parameters have been chosen as: pressure of $\unit[10]{Pa}$, external excitation $U\ix{rf}$ at $\unit[800]{V}$.
%
  \section{Secondary ion emission}
%  
  \section{Anion energy distributions in oxygen}
%
