%
\chapter*{Conclusions}\label{sec:chapter_conclusion}
\addchaptertocentry{Conclusion}
%
%
%   !!!WAS FEHLT: AUFGREIFEN DER HAUPTFRAGESTELLUNGEN AUS DER EINLEITUNG UND KLARE BEANTWORTUNG DER FORSCHUNGSFRAGEN, 
%   DIE DORT FORMULIERT WURDEN.
%   ICH WÜRDE DIE BEIDEN FRAGEN EXPLIZIT NENNEN UND DANN KLAR UND PRÄZISE BEANTWORTEN!!!
%
%
    In my thesis I applied a 1D and 2D Particle-in-Cell algorithm including Monte-Carlo-Collisions to simulate a cylindrical low-temperature ccrf discharge at low pressures of oxygen. The goal was to investigate the experimental findings from~\cite{Scheuer15}, where highly energetic anions were measured by a mass-spectrometer at the anode of a ccrf oxygen discharge. An explanation is proposed by Stoffels and Kawano et al.~\cite{Stoffels01,Kawano83}, suggesting that negative ions are produced by atomic or molecular neutral particles at the electrode surfaces.\\
    A 1D PIC model was used to understand the basic plasma physics and dynamics of negative ions including this surface process at the cathode. An estimate for the efficiency of secondary ion emission was made from experimental results~\Cite{Meichsner13}. I was able to reproduce qualitatively the energy distribution function of negative ions, which has a high energy peak due to the acceleration of secondary ions in the plasma sheath. A plateau and an additional low-energy peak in the EDF build up due to energy loss collisions in the bulk and sheath.\\
    However, I was not able to find anions from the cathode to impinge on the anode. This was due to the intrinsic symmetry of a 1D simulation. Hence a 2D simulation model was needed, in which asymmetry effects can occur and a self bias voltage can be applied.\\
    The 2D code was then validated by comparison with the 1D model. It could be shown that both simulations are in good agreement.\\
   In the 2D simulation asymmetrical driven discharges are studied, where cathode and anode configurations are varied and the self bias is introduced. Here, I found that the plasma adjusts self consistently to the different boundary configurations by re-shaping the sheaths and balancing particle currents for the global continuity condition. In case of a highly asymmetric discharge, where the plasma is in contact with a larger area of grounded walls, the sheaths are deformed to match this requirement. They are much larger in front of grounded walls than at a driven electrode. Also the decrease of bulk density and potential is more shallow there. This leads to the reduction of particle densities and velocities inside to balance the flux.\\
   Accordingly, the EDFs of the plasma species adopt and show the characteristic found in the experiment. Negative ions produced at the cathode can impinge on the opposing anode, because the self bias voltage offset creates an asymmetric acceleration, which lets the anions overcome the potential barrier in front of the grounded wall. The lower energy peak in the EDF of negative ions, which exists in 1D and is seen the experiment, disappears because absorption on the anode is considered.\\
    In summary, with respect to the scientific questions formulated in the beginning of this thesis, I could show that the physics and the dynamics of a ccrf low-pressure oxygen discharg is a complex balance of fast electrons and heavy positive and negative ions. The 1D PIC-MCC gave already a detailed understanding of the physics of this system. However, it lacks the possibility of asymmetries, which are necessary for realistic simulations of experiments. The 2D PIC-MCC code allows this kind of detailed comparison.\\
    This 2D PIC-MCC code gives detailed insight into the second research question, namely to understand the EDF of surface produced secondary particles, like negative oxygen ions. They gain a high energy by a strong acceleration in the sheath. The self bias voltage introduces an asymmetric acceleration of charged particles at the powered electrode. This results in a high particle count of negative ions at large kinetic energies as it is also observed experimentally at the grounded anode. Energy loss and charge exchange collisions lead to the broadening of this peak already in the plasma sheath, which creates a plateau over large ranges of energy. This can be seen in~\autoref{fig:twodedf_ni}, which is very close to the experimental measurements.
Therefore, the two major questions formulated at the beginning of the thesis could be addressed successfully.
%
    \par%
    In the future, the 2D PIC simulation can be used to study further aspects of asymmetric electronegative discharges, e.g. introducing complex sputter models including chemical reactions. This will also allow to apply it to industrial applications like etching for a more detailed microscopic description. Also, a self consistent approach for the self bias voltage may be implemented. One could measure the flux and accumulated charge locally and thereby calculate an individual dc voltage offset. The displacement current between sheath and bulk is used to approximate the self bias voltage in a balance of inward and outward charge current. Experiments might yield the relevant sheath capacities for the calculation of the self bias. Surface processes are still not very well studied theoretically and experimentally. If here more knowledge exist,  emission probabilities for surface processes from theory or experiment can be implemented.