\chapter{Physical properties of low temperature RF plasma}

  In this first chapter I will provide the necessary physical background 
  for this work about the numerical simulation of low temperature 
  capacitively coupled radio frequency plasma. Here both the mathematical 
  basics and method for the simulation, as well as the most important aspects
  about the plasma properties will be explained.

  \section{Plasma physics}

    \subsection{Capacitively coupled radio frequency plasma}

      The experiment where after the conducted simulation is modelled after resembles a capacitively coupled  radio frequency, low temperature plasma at low pressures of oxagen. \\
      Here, I will refer to a plasma as an globally quasi-neutral gas, consisting of freely moving charges --- e.g.\@ electrons, anions and kations --- and neutral gas particles. The ratio between charged and neutral species defines the \emph{degree of ionization}, which in this case is very low. The term of global neutrality emphasizes the purpose for different lenght scales inside the gas itself. Here, the associated condition $ n\ix{e} \,\, = \,\, n\ix{i}$ only is valid for areas larger than the so called \emph{Debye-sphere}. Inside this ball with a radius of $\lambda\ix{D}$, the \emph{Debye-length}, the afore-mentioned neutrality is not satisfied. A selection of the most important and basic physical properties and attributes have been compiled in~\ref{tabe:physicalquantities}. \\
      The creation of a plasma is accomplished by 2 parallel metal plates, the electrodes, where on at least one an alternating current at radio frequency is applied --- this kind of experimental setup is among the most common, thus being used for basic and in-depth studies of such plasma. Here a rf signal at exactly $\unit[13,56]{MHz}$ with an amplitude between $100$--$\unit[1000]{V}$ will be used. That said, a multitude of electric setups are possible, such as coated or grounded electrodes. Therefore, different regimes of operation ensue. At whole, the electrodes, neutral gas and electric layout resemble a dielectric hindered plate capacitor. This simplification can be used to access important physical properties, such as an additional voltage offset on one of the electrodes or charge currents at such. A basic scheme of a asymmetric rf discharge can be seen in~\ref{img:plasma scheme}. \\
      In the case of different electrode sizes, as seen in the afore-mentioned scheme, the potential inside the spatially restricted area between wall --- this can be also a grounded metal wall aside the electrodes --- and discharge can change drastically. This additional direct current offset is called \emph{self-bias} (see~\ref{subsec:self bias}). A dielectric displacement current between plasma sheath and volume accomodates as a result of the different time scales of particle movement. Especially, self-bias and displacement current play a key role in the following investigations, as a capacitive coupling between electrodes and power supply is difficult to model into a numerical kinetic simulation.\\

  \begin{table}[H]
    \centering
      \begin{tabular}{m{0.3\textwidth}|m{0.3\textwidth}|m{0.3\textwidth}}
        quantity & equation & relevance \\ 
        \hline  Debye-length &%
        $\lambda\ix{D,j}^2=\frac{\varepsilon\ix{0}k\ix{B}T\ix{j}}{n\ix{j}e^2}$ \newline%
        $\lambda\ix{D}^2=\left(\lambda\ix{D,e}^{-2}+\lambda\ix{D,I}^{-2}\right)^{-1}$ &%
        \\ 
        \hline plasma frequency &%
        $\omega\ix{P,j}^2=\frac{n\ix{j}e^2}{\varepsilon\ix{0}m\ix{j}}=\frac{v\ix{th,j}}{\lambda\ix{D,j}}=\frac{1}{\tau\ix{j}}$ &%
        \\ 
        \hline thermal velocity &%
        $v\ix{th,j}^2=\frac{k\ix{B}T\ix{j}}{m\ix{j}}$ &%
        \\   
        \hline mean particle distance &%
        $\overline{b}=\frac{\hbar}{m\ix{j}v\ix{th,j}}$ &%
        \\ 
        \hline Debye-Hückel potential &%
        $\Phi=\frac{Q}{4\pi\varepsilon|\vec{r}|}\euler^{-\frac{|\vec{r}|}{\lambda\ix{D}}}$ &%
        \\
        \hline
      \end{tabular}
    \caption{%
      }
    \label{tabe:physicalquantities}
  \end{table}
  
    \subsection{Sheath physics and wall interaction}

    \subsection{Self bias voltage}

    \subsection{Dielectric displacement current}

  \section{Negative ion physics}

    \subsection{Anion creation and distribution}

    \subsection{Dynamics and collisions}

  \section{Particle-In-Cell simulations with Monte Carlo-Colissions}

    \subsection{Principles}

    \subsection{2d3v PIC}

    \subsection{Monte Carlo-Collisions}
